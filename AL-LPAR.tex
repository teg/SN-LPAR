% Preempt some bad decisions by LLNCS
\RequirePackage{fixllncs}


\documentclass[orivec]{llncs}


% To get cases in equations
\usepackage{amsmath}

% To use definitions, lemmas and theorems.
%\usepackage{amsthm}

% twoheadleftarrow
\usepackage{amssymb}

% Nicer array and tabular layout (IEEE-suggested)
\usepackage{array}

% A more pleasant font
\usepackage[T1]{fontenc} % use postscript type 1 fonts
\usepackage{textcomp} % use symbols in TS1 encoding
%\usepackage[garamond]{mathdesign} % use a nice font

% A taste of french
%\usepackage[french]{babel}
\usepackage[utf8]{inputenc}

% Allow inclusion of graphics
%\usepackage[pdftex]{graphicx}

% To get url's
%\usepackage[hidelinks]{hyperref}

% Improves the text layout
\usepackage{microtype}

% To typeset derivations
\usepackage[lutzsyntax,pdftex]{virginialake}

% Landscape-orientated floats
\usepackage{rotating}

% To get semantic brackets
\usepackage{stmaryrd}

% Don't remember what this does
\usepackage[normalem]{ulem}

% Nice fractions
\usepackage{units}

% For commands with multiple optional arguments
\usepackage{xargs}

% Our own stuff
\usepackage{ALmacros}
%\usepackage{includeFigure}
%\input{ALpictures}
\usepackage{willemtools}

% Get counters and fonts right for theorem environments
\spnewtheorem{ALtheorem}{Theorem}{\bfseries}{\itshape}
\spnewtheorem{ALdefinition}[ALtheorem]{Definition}{\bfseries}{\upshape}
\spnewtheorem{ALproposition}[ALtheorem]{Proposition}{\bfseries}{\itshape}
\spnewtheorem{ALlemma}[ALtheorem]{Lemma}{\bfseries}{\itshape}


\newif\ifnonotes\nonotesfalse
\newcommand{\EMPTY}[1]{\ifnonotes\else{\color{red}    \noindent #1}\fi}
\newcommand{\Comment}[1]{\ifnonotes\else{\color{red}    \noindent{\bf Comment: }#1}\fi}
\newcommand{\Remark}[1]{\ifnonotes\else{\color{red}    \noindent{\bf Remark: }#1}\fi}
\newcommand{\TODO}[1]{\ifnonotes\else{\color{red}    \noindent{\bf TODO }#1}\fi}
%\newcommand{\Rem}[1]{\ifnonotes\else{\color{red}    \bigskip\\{\bf Remark: }#1 \bigskip\\}\fi}
%\nonotestrue % If this is commented, notes appear in the paper (if any)
%---------------------------------------------------------- REMOVE WHEN FINISHED



%-------------------------- FRONTMATTER

\title{A Proof of Strong Normalisation of the Typed Atomic Lambda-Calculus}

\author{Tom Gundersen\inst{1} \and
Willem Heijltjes\inst{2} \and
Michel Parigot\inst{1}}

\authorrunning{Gundersen, Heijltjes, \& Parigot}

\institute{Laboratoire Preuves, Programmes, Systèmes \\ CNRS \& Universit\'e Paris Diderot \\ {\tt teg@jklm.no, parigot@pps.univ-paris-diderot.fr}
\and University of Bath \\ {\tt w.b.heijltjes@bath.ac.uk} }


% ==============================================================================================================================================================================


\begin{document}


\maketitle

\begin{abstract}
The atomic lambda-calculus is a typed lambda-calculus with explicit sharing, which originates in a Curry-Howard interpretation of a deep-inference system for intuitionistic logic.
%
It has been shown that it allows fully lazy sharing to be reproduced in a typed setting.
%
In this paper we prove strong normalization of the typed atomic lambda-calculus using Tait's reducibility method.
\end{abstract}


% citations to preserve:
%	\cite{Dougherty-Lescanne-2001,Ghilezan-Ivetic-Lescanne-Likavec-2011} - intersection types for explicit substitution
%	\cite{Mellies-1995}
%




% =================================================================================================
\section{Introduction}



The \emph{atomic lambda-calculus} is a typeable lambda-calculus with explicit sharing, recently introduced in \cite{Gundersen-Heijltjes-Parigot-2013-JFLA,Gundersen-Heijltjes-Parigot-2013-LICS}, developed as the Curry--Howard interpretation of a deep-inference proof system for intuitionistic logic.
%
The present paper constitutes an important step in the development of its meta-theory, by extending intersection types and Tait's reducibility method to show strong normalisation of the typed atomic lambda-calculus.
%
The primary motivation for establishing this result is to demonstrate that the atomic lambda-calculus is a natural and well-behaved calculus, to which the main standard techniques and results apply.



\emph{Sharing} is an approach to efficient computation in lambda-calculi whereby duplication of subterms is deferred in favor of reference to a common, shared representation.
%
It is a leading principle behind, among others, explicit substitution calculi \cite{Abadi-Cardelli-Curien-Levy-1991,Lescanne-1994,David-Guillaume-2001,DiCosmo-Kesner-Polonovski-2003,Kesner-Lengrand-2007,Accattoli-Kesner-2010}, term calculi with strategies or higher-order transformations \cite{Hughes-1982,Ariola-Felleisen-Maraist-Odersky-Wadler-1995}, and sharing graphs in the style of Lamping \cite{Lamping-1990,Asperti-Guerrini-1998,VanOostrom-VanDeLooij-Zwitserlood-2004}.
%
The atomic lambda-calculus represents a novel category in this range.
%
As a typeable term calculus it is an alternative to explicit substitution calculi, providing a different perspective on sharing: as in sharing graphs, sharing is evaluated \emph{atomically}, by duplicating individual constructors.
%
A salient property is that the calculus implements \emph{fully lazy sharing} \cite{Wadsworth-1971,Hughes-1982,Balabonski-2012}, a degree of sharing that, while standard, had previously been achieved in lambda-calculi only by means of external transformations.



The paper \cite{Gundersen-Heijltjes-Parigot-2013-LICS} details how the atomic lambda-calculus and its sharing mechanisms are derived from \emph{deep inference} \cite{}, a proof methodology where inferences apply \emph{in context}, reminiscent of term rewriting.
%
Sharing in deep inference is by explicit \emph{contraction} rules, which implement atomic duplication by interacting with individual inferences.
%
By embedding natural deduction within the deep-inference formalism \emph{open deduction} \cite{Guglielmi-Gundersen-Parigot-2010}, duplication in traditional normalisation is broken up into atomic steps.
%
The atomic lambda-calculus is a direct computational interpretation of the resulting proof system.
%
The paper \cite{Gundersen-Heijltjes-Parigot-2013-LICS} further establishes the technical properties of full laziness and \emph{PSN}, preservation of strong normalisation with respect to the lambda-calculus.



In the present paper strong normalisation for the typed atomic lambda-calculus will be proven using the Tait-reducibility method \cite{Tait-1967,ProofsAndTypes}.
%
Reducibility is an abstract method compatible with higher-order logic, whose application provides a deeper understanding of reduction and its dynamics.
%
\emph{Intersection types} \cite{Coppo-DezaniCiancaglini-1980,Pottinger-1980,Krivine-1993} further allow the characterisation of exactly the strongly normalisable terms.
%
Together, the two form an important part of the meta-theory of a calculus (for example for explicit substitution calculi, in \cite{Dougherty-Lescanne-2001,Ghilezan-Ivetic-Lescanne-Likavec-2011}).



For compactness the proof will be carried out for a discipline of simple types, and will be shown extensible to intersection types.



% =================================================================================================
\section{The atomic lambda-calculus}
\label{sec:atomic-lambda-calculus}


The \emph{atomic lambda-calculus} introduced in \cite{Gundersen-Heijltjes-Parigot-2013-LICS} is a refined lambda-calculus, in which abstraction is split into a linear abstraction and a sharing operation.
%
Duplication and deletion proceeds locally through the evaluation of sharings.
%
The calculus consists of a \emph{basic calculus}, a standard linear lambda-calculus with a sharing construct, extended by a further construction called the \emph{distributor}.
%
The basic calculus is introduced first.



\begin{ALdefinition}
The \emph{basic calculus} $\Lambda_A^-$ is given by the grammar
%
\[
s,t,u,v,w  \quad\coloneq\quad x
	\mmid	\lambda x.t
	\mmid	(t)u
	\mmid	t\share xu
\]
%
where
(i) each variable may occur at most once,
(ii) in $\lambda x.t$ the variable $x$ must occur in $t$ and becomes bound, and
(iii) in $u\share xt$ each $x_i$ must occur in $u$ and becomes bound.
%
Terms of the basic calculus are called \emph{basic terms}.
\end{ALdefinition}
%
%
%
The four constructors are called \emph{variable}, \emph{abstraction}, \emph{application}, and \emph{sharing} respectively.
%
A nullary sharing $u\share*{}t$ is called a \emph{weakening}.



\begin{ALdefinition}
The \emph{atomic lambda-calculus} $\Lambda_a$ extends the basic calculus with the \emph{distributor} constructor.
%
The following mutually recursive grammars simultaneously define the \emph{atomic lambda terms} and the \emph{terms of multiplicity $n$}, or \emph{$n$-terms}, for every $n\geq0$, written $t^n$. 
%
\setMidspace{10pt}
\[
\begin{array}{@{}r@{}c@{}l@{~}l@{}}
		s,t,u,v,w \Coloneq & \ldots & \Mid u\distr xy{t^n}		& ^*
\\[5pt]		t^n   \Coloneq & \tup t & \Mid t^n\share[m]xu 		& ^{**}
\\[5pt]							   && \Mid t^n\distr[m]xy{t^m}	& ^{***}
\end{array}
\]
%
where 
(i) each variable may occur at most once, 
(ii) in $(^*)$ each variable $x_i$ must occur in $u$ and becomes bound, and $y$ must occur in $t^n$ and becomes bound, and 
(iii) in $(^{**})$ and $(^{***})$ each variable $x_i$ must occur in $t^n$ and becomes bound.
%
\end{ALdefinition}



The sharing and distributor constructors together will be referred to as \emph{closures} and abbreviated $\g$,$\g*$;
%
a sequence of closures $\g[1]\ldots\g[n]$ will be denoted $\g[i]_{i\leq n}$ or $\G$.
%
An $n$-term is of the form $\tup t\G$: an $n$-tuple of atomic lambda terms to which closures apply as though it were an atomic lambda term.
%
Where possible, terms and $n$-terms will not be distinguished, and both denoted $t,u,v$.
%
A sequence of variables $\vv x$ may be abbreviated $\vv*x$; a sharing is then denoted $\share*{\vv*x}t$.
%
Standard notions are: $\FV(u)$ is the set of free variables of $u$, and $u\sub xt$ denotes the substitution of $t$ for $x$ in $u$.
%
A series of substitutions $\sub{x_1}{t_1}\dotso\sub{x_n}{t_n}$ is abbreviated $\subn{x_i}{t_i}$.



For $t^n=\tup t\G$ let the \emph{$i^{\mbox{\scriptsize th}}$ projection} $\pi_i(t^n)$ be the atomic lambda term $t_i\G_i$ where $\G_i$ is obtained by removing the binders from $\G$ binding in any $t_j$ ($i\neq j$), and iteratively removing binders in $s_k$ when $x_k$ is removed from a distributor $\distr*{x_1,\dots,x_k,\dots,x_m}y{\tup s\G*}$.



Atomic lambda terms will be considered up to the congruence $(\sim)$ induced by~\eqref{eqn:explicit permutation} below;
note that due to linearity, both terms are only well-defined if both $\g$ and $\g*$ bind only in $t$.
%
\begin{equation}\label{eqn:explicit permutation}
	t\g\g* ~\sim~ t\g*\g
\end{equation}




The function $\tercoden-\,:\,\Lambda\to\Lambda_A$ interprets standard lambda-terms as atomic lambda terms.
%
And the function $\terden-\,:\,\Lambda_A\to\Lambda$ gives a denotation to atomic lambda-terms in terms of standard lambda-terms \cite{Gundersen-Heijltjes-Parigot-2013-LICS}.


% =================================================================================================
\subsection{Reduction rules}


Reduction in the atomic lambda-calculus, denoted $\rewrite$, consists of two parts:
\begin{itemize}
 \item linear $\beta$-reduction, denoted $\ALbeta$: the usual rule (rule~\ref{eqn:beta-reduction} below) applied linearly;
 \item sharing reductions, denoted $\ALnotbeta$, comprising two kinds of rule: (i) \emph{permutations} taking closures outward (rules~\ref{eqn:sharing above abstraction}--\ref{eqn:sharing above distribution}), and (ii) local \emph{transformations} that evaluate closures (rules~\ref{eqn:execute unary substitution}--\ref{eqn:distributor elimination}).
\end{itemize}


\noindent
{\bf Linear  $\beta$-reduction:}
%
\begin{equation}\tag{$\beta$}\label{eqn:beta-reduction}
	(\lambda x.u)t ~\ALbeta~ u\sub xt
\end{equation}


\noindent
{\bf Permutations of closures:}
%
\begin{align}\label{eqn:sharing above abstraction}
	\lambda x.t\g & ~\ALnotbeta~ (\lambda x.t)\g
		 & \text{if}~x\in\FV(t) &&&
\\\label{eqn:sharing above application function}
	(u\g)t & ~\ALnotbeta~ ((u)t)\g
\\\label{eqn:sharing above application argument}
	(u)t\g & ~\ALnotbeta~ ((u)t)\g
% \\\label{eqn:sharing above tuple}
% 	\tup*{t_1,\ldots,t_i\g,\ldots,t_n} & ~\ALnotbeta~ \tup t\g
\\\label{eqn:sharing above sharing}
	u\share x{t\g} & ~\ALnotbeta~ u\share xt\g
\\\label{eqn:sharing above distribution}
	u\distr xy{t\g} & ~\ALnotbeta~ u\distr xyt\g
		& \text{if}~y\in\FV(t) &&&
\end{align}



\noindent
{\bf Transformations on closures:}
%
\begin{gather}
\label{eqn:execute unary substitution}
	u\share*xt~\ALnotbeta~u\sub xt
\\[5pt]
\label{eqn:compound sharings}
	u\share*{\vv*x}{y_i}\share yt ~\ALnotbeta~
	u\share*{y_1,\dots,y_{i-1},\vv*x,y_{i+1},\dots,y_n}{t}
\\[5pt]
\label{eqn:share application}
	u\share x{(v)t} ~\ALnotbeta~
	u\subn{x_i}{(y_i)z_i}\share yv\share zt
\\[5pt]
\label{eqn:share abstraction}
	u\share x{\lambda x.t}~\ALnotbeta~ u\distr xx{\tup y\share yt}
\\[5pt]\notag
	u\distr xy{\tup t\share*{\vv*z}{y}}~\ALnotbeta~
	u\subn{x_i}{\lambda y_i.t_i\share*{\vv*z_i}{y_i}}
\\[5pt]
	\rule{4cm}{0pt}\mbox{where } \{\vv*z_i\} = \{\vv*z\}\cap\FV(t_i) \mbox{ for every }i\leq n
\label{eqn:distributor elimination}
\end{gather}

%
% \begin{multline}
% \label{eqn:distributor elimination 2}
% 	u\distr xy{\tup t}~\ALnotbeta~u\sub{x_1}{t'_1} \dotso \sub{x_j}{\lambda y.t_j} \dotso \sub{x_n}{t'_n} \\[5pt]
% 			\mbox{where } y\in\FV(t_j) \mbox{ and } t'_i = \lambda y_i.t_i\share*{}{y_i}
% \end{multline}

\begin{remark}
Rule~\eqref{eqn:execute unary substitution}, which evaluates unary sharings by actual substitutions, is redundant and may be considered an optimisation: its functionality is subsumed by the rules for $n$-ary sharings.
\end{remark}
%
%\begin{remark}
%Compared to the presentation in \cite{Gundersen-Heijltjes-Parigot-2013-JFLA,Gundersen-Heijltjes-Parigot-2013-LICS} the rewrite rule to lift closures out of a tuple has been dropped (rule 5 in op.cit.), for the combined reason that it was redundant and it complicated the strong normalisation proof---see also the start of Section~\ref{sec:SNproof}.
%\end{remark}


The fact that a term $u$ reduces to $v$ in exactly $n$ steps will be denoted $u \rewrite^n v$, while an arbitrary number of steps is indicated simply by $\rewrite$.
%
A term $u$ is called \emph{strongly normalisable} if all the reductions sequences starting with $u$ are finite.
%
The set of strongly normalisable terms is denoted $\SN$.
%
Reduction in the atomic lambda-calculus commutes 1--1 with substitution, due to the linearity condition on free variables.


\begin{ALlemma}
For atomic lambda-terms $u$, $u'$, $v$ and $v'$ and variable $x \in \FV(u)$,
%
if $u \rewrite^1 u'$, then $u\sub xv \rewrite^1 u'\sub xv$; and
if $v\rewrite^1 v'$, then $u\sub xv \rewrite^1 u\sub x{v'}$.
\end{ALlemma}

% =================================================================================================
\subsection{Basic properties of the atomic lambda-calculus}
\label{ssec:basic properties}

We collect in this section the main basic properties we are using in the strong normalisation proof. The two main properties are (i) the strong normalisation property of the sharing reduction, and (ii) the  decomposition of the computational content of sharings and distributors.

\begin{ALtheorem}[{\cite[Theorem 11]{Gundersen-Heijltjes-Parigot-2013-LICS}}]
The reduction $\ALnotbeta$ is strongly normalising and confluent.
\end{ALtheorem}

\noindent
The \emph{unfolding} $\unfold(t)$ of an atomic lambda-term $t$ is its normal form under $\ALnotbeta$.
%
It is of the form $u\G$ where $\G$ are sharings $\share*{\vv*x}y$ of the free variables $y$ in $t$ that occur in shared subterms.
%
Sharing in $u$ occurs only as $\lambda y.v\share*{\vv*x}y$, of bound variables immediately within the scope of their binder, and no distributors occur.
%
%
\begin{ALdefinition}
The \emph{unfolded body} $\sn(t)$ of $t$ is the largest subterm of $\unfold(t)$ not of the form $u\g$.
\end{ALdefinition}
%
%
%For $t$ with unfolding $s$, let $\sn(t)$ denote the largest subterm of $s$ not of the form $u\g$; then $s$ is of the form $\sn(t)\share*{x_{1,1},\dots,x_{1,m_1}}{x_1}\dots\share*{x_{n,1},\dots,x_{n,m_n}}{x_n}$ (with the $x_{i,j}$ and $x_k$ distinct).
%
%
The unfolded body of a term is what is duplicated during reduction.
%
To identify the various copies, let a \emph{variant} of a term $t$ be any term obtained form $t$ by renaming certain (bound or free) variables.
%
A variant is \emph{fresh} if all its variables are fresh, and $t^i$ is the fresh variant of $t$ obtained by replacing each
variable $x$ by a fresh variable $x^i$.
%
The following basic facts then characterise $\sn$.


\begin{ALproposition}
\[
\begin{aligned}
	\sn(x) = x \quads4
	\sn(\lambda x.t) & =\lambda x.\sn(t) \quads4
	\sn((u)v) = (\sn(u))\sn(v)
\\[5pt]
   \sn(u\share xt) & =\sn(u)\subn{x_i}{\sn(t)^i}
\\[5pt]
	\sn(u\distr xyt) &=\sn(u)\subn{x_i}{\sn(\lambda y.\pi_i(t))^i}
\end{aligned}
\]
\end{ALproposition}


\begin{ALproposition}\label{prop:sn_pi}
For $t=\tup t\G$, $\sn(\pi_i(t))=\sn(t_i\share*{}{x_1}\dots\share*{}{x_m}\G)$ where
$\vv[m]x$ are the free variables of all $t_j$ ($i\neq j$).
\end{ALproposition}


%For $\unfold(t)=\sn(t)\share*{\vv*x_j}{y_j}_{j\leq m}$, to characterise the effects of duplication on the free variables $y_j$ of a term, let $[\sh(t):1,\dotsc,n]$ denote the sharings $\share*{\vv*x_j^{\;i}}{y_j}_{j\leq m,i\leq n}$, abbreviated $[\sh (t)]$ where possible.
%
To characterise the effects of duplication on the free variables of a term $t$, let $\FV(t)=\{y_1,\dots,y_k\}$ and $\FV(\sn(t))=\{\vv*y_1,\dotsc,\vv*y_k\}$, and define the \emph{renamings} of an $n$-fold duplication of $t$ by $[\sh (t):1,\dots,n] = \share*{\vv*y_i^{\;1},\dots,\vv*y_i^{\;n}}{y_i}_{i\leq k}$ (abbreviated $[\sh(t)]$ where possible).
%
The unfolded body and the renamings give the following key decomposition properties of the computational content of closures.
%
%
%
%Next, for any term $t$, with $\FV(t)=\{y_1,\dots,y_k\}$ and $\FV(\sn(t))=\{y_{1,1},\dots,y_{k,m_k}\}$, we define
%$[\sh (t):1,\dots,n] =
%\share*{y_{1,1}^1,\dots,y_{1,m_1}^n}{y_1}\dots
%\share*{y_{k,1}^1,\dots,y_{k,m_k}^n}{y_k}$.
%%
%When there is no ambiguity, we will use the notation $[\sh (t)]$ in place of $[\sh (t):1,\dots,n]$.
%
%The two following lemmas then show key decomposition properties of the computational content of sharings and distributors.


\begin{ALlemma}
\label{lem:unsharing}
%A term $u\share xt$ reduces to $u\subn{x_i}{\sn(t)^i}[\sh (t):1,\dots,n]$.
$u\share xt \rewrite u\subn{x_i}{\sn(t)^i}[\sh (t):1,\dots,n]$
\end{ALlemma}

\begin{ALlemma}
\label{lem:undist}
%A term $u\distr xyt$ reduces to $u\subn{x_i}{\sn(\lambda y.\pi_i(t))^i}[\sh (\lambda y.t):1,\dots,n]$.
$u\distr xyt \rewrite u\subn{x_i}{\sn(\lambda y.\pi_i(t))^i}[\sh (\lambda y.t):1,\dots,n]$
\end{ALlemma}




% =================================================================================================
\section{Typed atomic lambda-calculus}\label{sec:types}


The simply typed atomic lambda-calculus $S_a$ is defined by the following rules (see \cite{Gundersen-Heijltjes-Parigot-2013-JFLA,Gundersen-Heijltjes-Parigot-2013-LICS}).
%
%The atomic lambda-calculus with intersection types $D_a$ extends $S_a$ by the rules indicated below.
%
As for the pure atomic lambda-calculus we derive two sorts of expressions:
\begin{itemize}

 \item Judgments for atomic lambda-terms $x_1: A_1,\dots, x_n : A_n \vdash t : B$ where $t$ is a term, $A_1, \dots, A_n, B$ are formulas built with $\vlim$, and $\vv x$ are the free variables of $t$;

 \item Judgments for terms of multiplicity $k>0$, $x_1: A_1,\dots, x_n : A_n \vdash t^k : B_1 \vlan\cdots\vlan B_k$, where $t$ is a term of multiplicity $k$, $A_1, \dots, A_n, B_1, \dots, B_k$ are formulas built with $\vlim$, and $\vv x$ are the free variables of $t^k$.

\end{itemize}
Antecedents $x_1: A_1,\dots, x_n : A_n$ of judgements are treated as sets and will be abbreviated $\Gamma$ or $\Delta$. In addition $x_1\ldots x_n : B$ abbreviates $x_1\colon B,\ldots,x_n : B$.
%

\bigskip
\noindent
{\bf Typing rules of $S_a$ for atomic lambda-terms:}
\[
\begin{array}{@{}c@{}}
	\begin{array}{c@{\quads4}c}
	  \vlinf{}{\typeabs}
	   {\Gamma \vdash \lambda x.t : A \vlim B}
	   {\Gamma, x : A \vdash t : B}
	&
	  \vliiinf{}{\typeapp}
	   {\Gamma, \Delta \vdash (u)v : B}
	   {\Gamma \vdash u : A\vlim B}
	   {}
	   {\Delta \vdash v : A}
	\\ \\
	  \vlinf{}{\typevar}{x:A\vdash x:A}{}
	&
	  \vliiinf{}{\typeshare}
	   {\Gamma, \Delta \vdash u\share xt : A}
	   {\Gamma, x_1\dots x_n : B \vdash u : A}
	   {}
	   {\Delta \vdash t : B}
	\end{array}
\\ \\
  \vliiinf{}{\typedist}
   {\Gamma, \Delta \vdash s\distr xy{t^n} : C}
   {\Gamma, x_1 : A \vlim B_1, \dots, x_n : A \vlim B_n \vdash s : C}
   {}
   {\Delta, y : A \vdash t^n : B_1\vlan\cdots\vlan B_n}
%\\ \\ \\
\end{array}
\]

\bigskip
\noindent
{\bf  Typing rules of $S_a$ for terms of multiplicity {\boldmath $k$}:}
\[
\begin{array}{@{}c@{}}
  \vliiinf{}{\typetuple _k}
   {\Gamma_1, \dots, \Gamma_k \vdash \langle t_1,\ldots,t_k\rangle : A_1\vlan\cdots\vlan A_k}
   {\Gamma_1 \vdash t_1 : A_1}
   {\quad\cdots\quad}
   {\Gamma_k \vdash t_k : A_k}
\\ \\
  \vliiinf{}{\typeshare _k}
   {\Gamma, \Delta \vdash u^k\share xt : A_1\vlan\cdots\vlan A_k}
   {\Gamma, x_1\dots x_n : B \vdash u^k : A_1\vlan\cdots\vlan A_k}
   {}
   {\Delta \vdash t : B}
\\ \\
  \vliiinf{}{\typedist _k}
   {\Gamma, \Delta \vdash u^k\distr xy{t^n} : C_1\vlan\cdots\vlan C_k}
   {\Gamma, x_1 : A \vlim B_1, \dots, x_n : A \vlim B_n \vdash u^k : C_1\vlan\cdots\vlan C_k}
   {}
   {\Delta, y : A \vdash t^n : B_1\vlan\cdots\vlan B_n}
\end{array}
\]

%\bigskip
%\noindent
%{\bf  Typing rules of $D_a$}
%\[
%\begin{array}{c}
%	  \vliiinf{}{\cap_i}
%	   {\Gamma\vdash u : A\cap B}
%	   {\Gamma \vdash u : A}
%	   {}
%	   {\Gamma \vdash u : B}
%\quads4
%         \vlinf{}{\cap_{e1}}
%	   {\Gamma \vdash u : A}
%	   {\Gamma \vdash  u : A\cap B}
%\quads2
%         \vlinf{}{\cap_{e2}}
%	   {\Gamma \vdash   u : B}
%	   {\Gamma \vdash u : A\cap B}
%\end{array}
%\]

The type system $S_a$ of the atomic lambda-calculus is a refinement of the simply typed lambda-calculus $S$: the rule $\typevar$, $\typeabs$ and $\typeapp$ are the rules of $S$ restricted by the linearity condition. Two rules for sharing and distributor are added, which are some kind of cuts, the one for sharing being natural and the one for distributor a bit more involved.


The typed atomic lambda-calculus enjoys the usual properties we are expecting from typed systems, in particular subject reduction, which holds both for ${S_a}$ and  ${S_a}$.


\begin{ALtheorem}[\cite{Gundersen-Heijltjes-Parigot-2013-JFLA}]
If $\Gamma\vdash u : A$ and $u \rewrite v$, then $\Gamma\vdash v : A$.
\end{ALtheorem}

\noindent
Morever,
%both simple types and intersection types
types are preserved in the interpretation of standard lambda-terms as atomic lambda-terms.

\begin{ALproposition}\label{prop:types preserved}
If $\Gamma\vdash_S N:A$, then $\Gamma\vdash_{S_a} \tercoden N:A$
\end{ALproposition}

%\begin{ALproposition}\label{prop:intersection types preserved} $\;$\\
%(i) If $\Gamma\vdash_S N:A$, then $\Gamma\vdash_{S_a} \tercoden N:A$ \\
%(ii) If $\Gamma\vdash_D N:A$, then $\Gamma\vdash_{D_a} \tercoden N:A$
%\end{ALproposition}

% \begin{proof}
% %By induction on the typing derivation for $\Gamma\vdash N:A$.
% %
% %The case where consists of a single axiom is immediate, and if the last rule in the derivation is one of $\typeapp$, $\cap_{e1}$, $\cap_{e2}$, or $\cap_i$, the induction hypothesis applies immediately.
% %
% A typing derivation for $\Gamma \vdash N : A$ translates directly to one for $\Gamma\vdash \tercoden N : A$, with the following notes:
% %
% 1) the derivation for $\tercoden N$ ends in an additional series of $(\typeshare)$-inferences as displayed below left; and
% %
% 2) a $(\lambda)$-inference in the derivation for $N$ translates to a consecutive $(\lambda)$-inference and $(\typeshare)$-inference, as illustrated below right.
% %
% \[
% 	{\vliiinf{}{\typeshare}
% 	   {\Gamma, x_i : B  \vdash u\share*{x^1_i,\dotsc,x^n_i}{x_i} : A}
% 	   {\Gamma, x^1_i \dots x^n_i : B \vdash u : A }
% 	   {}
% 	   {x_i : B \vdash x_i : B }
% 	}
% \qquad
% 	\vlderivation{
% 	 \vlin{}{\lambda}
% 	  {\Gamma \vdash \lambda x.u\share*{x^1,\dotsc,x^n}x : A\to B}
% 	  {\vliiin{}{\typeshare}
% 	   {\Gamma, x : B  \vdash u\share*{x^1,\dotsc,x^n}x : A}
% 	   {\vlhy{ \Gamma, x^1 \dots x^n : B \vdash u : A }}
% 	   {\vlhy{\ }}
% 	   {\vlhy{ x : B \vdash x : B }}
% 	  }
% 	}
% \quad
% \]
% \end{proof}




\begin{proof}
%
A typing derivation for $\Gamma \vdash N : A$ translates directly to one for $\Gamma\vdash \tercoden N : A$, with $(\typeshare)$-inferences added where required.
\end{proof}
%the following notes:
%%
%1) the derivation for $\tercoden N$ ends in an additional series of combined $(\typeshare)$-inferences and axioms, as displayed below left; and
%%
%2) a $(\lambda)$-inference in the derivation for $N$ translates to a combined $(\lambda)$-inference, $(\typeshare)$-inference, and axiom, as illustrated below right.
%%
%\[
%    \vlderivation{
%     {\vliiin{}{\typeshare}
%       {\Gamma, x_i : B  \vdash u\share*{x^1_i,\dotsc,x^n_i}{x_i} : A}
%       {\vlhy{\Gamma, x^1_i \dots x^n_i : B \vdash u : A }}
%       {\vlhy{\ }}
%       {\vlin{}{\typevar}{x_i : B \vdash x_i : B }{\vlhy{\ }}}
%    }}
%\qquad
%    \vlderivation{
%     \vlin{}{\lambda}
%      {\Gamma \vdash \lambda x.u\share*{x^1,\dotsc,x^n}x : A\to B}
%      {\vliiin{}{\typeshare}
%       {\Gamma, x : B  \vdash u\share*{x^1,\dotsc,x^n}x : A}
%       {\vlhy{ \Gamma, x^1 \dots x^n : B \vdash u : A }}
%       {\vlhy{\ }}
%       {\vlin{}{\typevar}{x : B \vdash x : B}{\vlhy{\ }}}
%      }
%    }
%\quad
%\]
%\end{proof}


% =================================================================================================
\section{Proof of Strong Normalisation}\label{sec:SNproof}

In this section we prove the strong normalisation theorem for atomic lambda-terms, typed in the system $S_a$,
% and $D_a$
using Tait's reducibility method. The proof of the main proposition (Proposition~\ref{prop:IntSubst}) relies on closure properties of the reducibility sets (Lemma~\ref{lem:IntCaseLambdaSharing}), which again relies on closure properties on the set of strongly normalisable atomic lambda trems proved in the next section (Section~\ref{sec:ClosPropSN}).

For simplifying the presentation, we consider in the following that terms of multiplicity $n$ are of the form $\langle t^1,\dots, t^n \rangle\G$ with $t$ in unfolding. This means in particular that we do not compute inside $\langle ... \rangle$. This
restriction is  natural in the context of sharing calculus and all the useful computation strategies satisfy it, in particular the one reproducing fully lazy sharing.


We denote by $\VAR$ the set of variables.
For each term $u \in \SN$, we denote $S(u)$ the sum of the lengths of the reduction sequences of $u$ leading to a normal form. The length $lh(t)$ of a term $t$ is defined in the usual way, except that in a distributor, the lenght of its body $\lambda y.\langle t^1,\dots, t^n \rangle\G$ is defined as the length of $t\G$.
We denote by $\overline{w[\Gamma]}$ a (possibly empty) sequence $w_1[\Gamma_1]\dots w_n[\Gamma_n]$, where $w_1,\dots ,w_n$ are terms (we suppose that a non-empty sequence  $\overline{w[\Gamma]}$ has at least one term $w_1$).
By $(u)\overline{w[\Gamma]}$, we mean $((...(((u)w_1)[\Gamma_1])\dots )w_n)[\Gamma_n]$; thus, each $\G[i]$ may bind not only in $w_i$, but also in $u$, and in all $\G[j]$ and $w_j$ for $j<i$.


\begin{ALdefinition}
The \emph{value} $|A|$ of a formula $A$ is defined inductively by:
\[
\begin{array}{lll}
|X| & = & \mathcal{N}
\\
%|A\cap B| & = & |A| \cap |B|
%\\
|A\vlim B| & = & \mbox{\{$u\in\Lambda_a$ \;|\; for each term $v\in |A|$, $(u)v \in |B|$ \}}
\\
\end{array}
\]

The value of a conjunctive formula is defined as
\[
|A_1\vlan\dots\vlan A_n| \; = \; \mbox{\{$t^n$ \;|\; for each $i\leq n$, $\pi_i(t) \in |A_i|$; and for any $x$, $x\G\in\SN$ \} ,}
\]
where $\G$ is the sequence of all the weakenings in $t^n$.

Values of formulas are called {\em reducibility sets}. \\
\end{ALdefinition}


\begin{remark}
If $t\in|A|$ and $t'$ is a variant of $t$, then $t'\in|A|$.
\end{remark}

\begin{ALlemma}\label{lem:HeadVar}
If $x\in \VAR$ and $\overline{t} \in \SN^{<\omega}$, then
$(x)\overline{t} \in \SN$.
\end{ALlemma}

\begin{ALproposition}\label{prop:IntBase}
For each formula $A$, $\VAR \subseteq |A| \subseteq  \SN$
\end{ALproposition}

\begin{ALproposition}
 For any formulas $A_1,\dots,A_n$, $|A_1\vlan\dots\vlan A_n| \subseteq \SN$.
\end{ALproposition}

\begin{proof}
The result follows directly from the defintion of  $|A_1\vlan\dots\vlan A_n|$ and Proposition~\ref{prop:IntBase}.
\end{proof}

\begin{ALlemma}\label{lem:RedStab}
For any formula $A$,  if $u\in |A|$ and $u \rightsquigarrow v$, then $v \in |A|$ .
\end{ALlemma}

\begin{proof}
 Immediate by induction on $A$.
\end{proof}

\begin{ALlemma}\label{lem:Red-AddSharings}
For any formula $B$,  if $(u)\overline{w}\in |B|$ then $(u\share xx)\overline{w} \in |B|$ .
\end{ALlemma}

\begin{ALlemma}\label{lem:IntCaseLambdaSharing} $\;$ \\
(i) If $(u\{v/x\})\overline{w} \in |B|$ and $v\in\SN$, then $((\lambda x.u) v)\overline{w} \in |B|$.
\\
(ii) If $(u\{\sn(t)^1/x_1,\dots,\sn(t)^n/x_n\}[\sh (t)])\overline{w} \in |B|$ and $t\in\SN$, \\ then  $(u[x_1,\dots,x_n \leftarrow t])\overline{w} \in |B|$.
\\
(iii) If $(u\{\sn(\lambda y.t[\Gamma])^1/x_1,\dots,\sn(\lambda y.t[\Gamma])^n/x_n\}[\sh (\lambda y.t[\Gamma])])\overline{w} \in |B|$ and $t[\Gamma]\in\SN$, \\ then  $(u[x_1,\dots,x_n \twoheadleftarrow \lambda y\langle t^1,\dots,t^n \rangle[\Gamma]])\overline{w} \in |B|$.
\end{ALlemma}

\begin{proof}
The first statement follows from Lemma~\ref{lem:IntCaseLambda0}, and the full proof can be found in the Appendix.
We prove only the second statement here, the proof of the third being similar.
We proceed by induction on $B$.
\\
1) $B$ is a variable. Then $|B| = \SN$ and the result is given by Lemma~\ref{lem:IntCaseSharing0}, which has been delayed to the next section.
\\
2) $B= C\vlim D$.
Suppose $(u\{\sn(t)^1/x_1,\dots,\sn(t)^n/x_n\}[\sh (t)])\overline{w} \in |C\vlim D|$ and $t\in\SN$. Let $v\in|C|$. We prove that  $((u[x_1,\dots,x_n \leftarrow t])\overline{w})v \in |D|$. Because $(u\{\sn(t)^1/x_1,\dots,\sn(t)^n/x_n\}[\sh (t)])\overline{w} \in |C\vlim D|$ and $v\in|C|$, we have $((u\{\sn(t)^1/x_1,\dots,\sn(t)^n/x_n\}[\sh (t)])\overline{w})v \in |D|$ and by induction hypothesis, $((u[x_1,\dots,x_n \leftarrow t])\overline{w})v \in |D|$. It follows that $(u[x_1,\dots,x_n \leftarrow t])\overline{w} \in |C\vlim D|$.
%\\
%3) $B= C\cap B$.
%The result directly follows from the induction hypothesis using the fact that $|C\cap D| = |C|\cap|D|$.

\end{proof}

\begin{ALproposition}\label{prop:IntSubst}
If $x_1:A_1,\ldots, x_n:A_n \vdash_S u:B$ and $v_1 \in |A_1|, \dots, v_n \in |A_n|$, then $u\{v_1/x_1, \dots, v_n/x_n\} \in |B|$.
\end{ALproposition}

\begin{proof}

We proceed by induction on the derivation of $x_1:A_1,\ldots, x_n:A_n \vdash_S u:B$.
%
We consider the last rule of the derivation.
%
We omit the rules for terms of multiplicity $k$, which can be treated similarly to the rules we do consider, and also the rules for intersection types, whose treatment is immediate.
\medskip\\
1) The last rule is  $$\vlinf{}{\typevar}{x:A\vdash x:A}{}$$
Let $v\in |A|$. In this case we have $x\{v/x\} = v\in |A|$
\medskip\\
2) The last rule is
\[
 \vliiinf{}{\typeapp}
	   {x_1:C_1,\dots,x_n:C_n,y_1:D_1,\dots,y_m:D_m\vdash (u)v : B}
	   {x_1:C_1,\dots,x_n:C_n\vdash u : A\vlim B}
	   {}
	   {y_1:D_1,\dots,y_m:D_m\ \vdash v : A}
\]
Let $v_1 \in |C_1|,\dots,v_n \in |C_n|,w_1 \in |D_1|,\dots,w_m \in |D_m|$.  We have
\begin{multline*}
((u)v)\{v_1/x_1,\dots,v_n/x_n,w_1/y_1,\dots,w_m/y_m\} =\\
(u\{v_1/x_1,\dots,v_n/x_n\})v\{w_1/y_1,\dots,w_m/y_m\}\;.
\end{multline*}
By induction hypothesis,
\begin{multline*}
u\{v_1/x_1,\dots,v_n/x_n\}\in |A\vlim B|\;\mbox{and}\;v\{w_1/y_1,\dots,w_m/y_m\}\in |A|\; \\
\mbox{and therefore}\;(u\{v_1/x_1,\dots,v_n/x_n\})v\{w_1/y_1,\dots,w_m/y_m\}\in |B|\;.
\end{multline*}
\medskip\\
3) The last rule is
\[
 \vlinf{}{\typeabs}
	   {x_1:C_1,\dots,x_n:C_n \vdash \lambda x.t : A \vlim B}
	   {x_1:C_1,\dots,x_n:C_n, x : A \vdash t : B}
\]
Let $v_1 \in |C_1|,\dots,v_n \in |C_n|$. We have
\[(\lambda x.t)\{v_1/x_1,\dots,v_n/x_n\} = \lambda x.t\{v_1/x_1,\dots,v_n/x_n\}\;.\]
Let $w\in |A|$. By induction hypothesis we have $t\{v_1/x_1,\dots,v_n/x_n, w/x\}\in |B|$ and therefore $t\{v_1/x_1,\dots,v_n/x_n\}\{ w/x\}\in |B|$. By Lemma~\ref{lem:IntCaseLambdaSharing}, it follows  $(\lambda x.t\{v_1/x_1,\dots,v_n/x_n\})w \in |B|$. Hence  $(\lambda x.t)\{v_1/x_1,\dots,v_n/x_n\}\in | A\vlim B|$.
\medskip\\
4) The last rule is
\[
 	  \vliiinf{}{\typeshare}
	   {y_1:C_1,\dots,y_k:C_k, z_1:D_1,\dots,z_m:D_m \vdash u\share xt : A}
	   {y_1:C_1,\dots,y_k:C_k, x_1:B,\dots, x_n : B \vdash u : A}
	   {}
	   {z_1:D_1,\dots,z_m:D_m \vdash t : B}
\]
\\
Let $v_1 \in |C_1|,\dots,v_k \in |C_k|,w_1 \in |D_1|,\dots,w_m \in |D_m|$.  We have to prove
\\
$ (u[x_1,\dots, x_n \leftarrow t])\{v_1/y_1,\dots,v_k/y_k,w_1/z_1,\dots,w_m/z_m\} \in |A|$, i.e.
\\
$ u \{v_1/y_1,\dots,v_k/y_k\} [x_1,\dots, x_n \leftarrow t\{w_1/z_1,\dots,w_m/z_m\} ] \in |A|$.
\\
By induction hypothesis, we have $t' = t\{w_1/z_1,\dots,w_m/z_m\} \in|B|$. Let $\sn (t')^1, \dots, \sn (t')^n$ variants of $\sn (t')$ with fresh variables. Because $t'\in |B|$, the unfolding of $t'$ belongs also to $|B|$. It follows that $\sn (t') \in |B|$ and also $\sn (t')^1 \in |B|, \dots, \sn (t')^n \in |B|$. Therefore by induction hypothesis
$ u\{v_1/y_1,\dots,v_k/y_k\} \{\sn (t')^1/x_1,\dots, \sn (t')^n/x_n\} \in |A|$ and by Lemma~\ref{lem:Red-AddSharings}, $ u\{v_1/y_1,\dots,v_k/y_k\} \{\sn (t')^1/x_1,\dots, \sn (t')^n/x_n\}[\sh (t)] \in |A|$ . It follows by Lemma~\ref{lem:IntCaseLambdaSharing} that
\\
$ u\{v_1/y_1,\dots,v_k/y_k\} [x_1,\dots, x_n \leftarrow t'] \in |A|$, i.e.
\\
$ u \{v_1/y_1,\dots,v_k/y_k\} [x_1,\dots, x_n \leftarrow t\{w_1/z_1,\dots,w_m/z_m\} ] \in |A|$.
\medskip\\
5) The last rule is
\[
   \vliiinf{}{\typedist}
   {y_1:C_1,\dots,y_k:C_k, z_1:D_1,\dots,z_m:D_m \vdash s\distr xy{u} : C}
   {y_1:C_1,\dots,y_k:C_k, x_1, \dots, x_n : A \vlim B \vdash s : C}
   {}
   {z_1:D_1,\dots,z_m:D_m, y : A \vdash u : B\vlan\cdots\vlan B}
\]
\\
Let $v_1 \in |C_1|,\dots,v_k \in |C_k|,w_1 \in |D_1|,\dots,w_m \in |D_m|$.  We have to prove
$ (s[x_1,\dots, x_n \twoheadleftarrow \lambda y.u])\{v_1/y_1,\dots,v_k/y_k,w_1/z_1,\dots,w_m/z_m\} \in |C|$, i.e.
\\
$ s\{v_1/y_1,\dots,v_k/y_k\} [x_1,\dots, x_n \twoheadleftarrow \lambda y.u\{w_1/z_1,\dots,w_m/z_m\} ] \in |C|$.
\\
The term  $u$ is of the form $\langle t^1,\dots, t^n \rangle [\Gamma]$, with $t^1,\dots, t^n$ fresh variant of $t$ in unfolding. By induction hypothesis, we have, for each $r\in |A|$, $\langle t^1,\dots, t^n \rangle [\Gamma]\{w_1/z_1,\dots,w_m/z_m, r/y\} \in|B\vlan\cdots\vlan B|$ and therefore $t[\Gamma]\{w_1/z_1,\dots,w_m/z_m, r/y \} \in|B|$.
By Lemma~\ref{lem:IntCaseLambdaSharing}, it follows that, for each $r\in |A|$, $(\lambda y.t[\Gamma]\{w_1/z_1,\dots,w_m/z_m \})r \in|B|$ and therefore by definition of $|-|$, $\lambda y.t[\Gamma]\{w_1/z_1,\dots,w_m/z_m \} \in|A \vlim B|$. \\
Let  $t' = t[\Gamma]\{w_1/z_1,\dots,w_m/z_m\}$. Because $\lambda y.t'\in |A \vlim B|$, the unfolding of $\lambda y.t'$ belongs also to $|A \vlim B|$, by lemma~\ref{lem:RedStab}. It follows that $\sn (\lambda y.t') \in |A \vlim B|$ and also $\sn (\lambda y.t')^1 \in |A \vlim B|, \dots, \sn (\lambda y.t')^n \in |A \vlim B|$. Therefore
$ s  \{v_1/y_1,\dots,v_k/y_k\}  \{\sn (\lambda y.t')^1/x_1,\dots, \sn (\lambda y.t')^n/x_n\} \in |C|$ and by Lemma~\ref{lem:Red-AddSharings},
\[
s  \{v_1/y_1,\dots,v_k/y_k\}  \{\sn (\lambda y.t')^1/x_1,\dots, \sn (\lambda y.t')^n/x_n\}[\sh (\lambda y.t')] \in |C|\;.
\]
It follows by Lemma~\ref{lem:IntCaseLambdaSharing} that
\[s \{v_1/y_1,\dots,v_k/y_k\}[x_1,\dots, x_n \twoheadleftarrow \lambda y.t'] \in |C|\;,\;\mbox{i.e.}\]
\[
s\{v_1/y_1,\dots,v_k/y_k\} [x_1,\dots, x_n \twoheadleftarrow \lambda y.u\{w_1/z_1,\dots,w_m/z_m\} ] \in |C|\;.
\]
\end{proof}


\begin{ALtheorem}\label{thm:SN}
If $x_1:A_1,\dots, x_n:A_n \vdash_S u:B$ then $u \in \SN$.
\end{ALtheorem}



\begin{proof}

Suppose $x_1:A_1,\dots, x_n:A_n \vdash_S u:B$. By Proposition~\ref{prop:IntBase}, we have $x_1\in |A_1|,\dots, x_n \in|A_n|$. Therefore by Proposition~\ref{prop:IntSubst},  $u\{x_1/x_1, \dots, x_n/x_n\} \in |B|$, i.e. $u \in |B|$. Therefore by Proposition~\ref{prop:IntBase}, we have $u \in \SN$.

\end{proof}

% =================================================================================================



\section{Closure properties of Strongly Normalisable Atomic Lambda Terms}\label{sec:ClosPropSN}


In this section we prove closure properties for the set of strongly normalisable atomic lambda terms
which are used in Section 4.

\begin{ALlemma}\label{lem:IntCaseLambda0}
If $((u\{v/x\})[\Gamma][\Gamma'])\overline{w[\Gamma]} \in \SN$, then $((((\lambda x.u)[\Gamma]) v)[\Gamma'])\overline{w[\Gamma]} \in \SN$.
\end{ALlemma}

\begin{proof}
 Let $T = ((((\lambda x.u)[\Gamma]) v)[\Gamma'])\overline{w[\Gamma]}$ and $T' = ((u[v/x])[\Gamma][\Gamma'])\overline{w[\Gamma]}$.
\\
In order to show that $T\in\SN$, we show that each one-step reduction from $T$ gives a term $U\in\SN$.
%
We proceed by induction on $Mes(T) = (S(T'), lh(u), lh(v)), lh([\gamma])$ where
 $lh([\Gamma])$ is the length of the sequence $[\Gamma]$.
\end{proof}

\newcommand{\term}{{\mathsf{t}}}

To each closure $[\gamma]$, we associate its term $t(\gamma)$ and its computation $[\gamma]^v $ defined as follows:
\begin{itemize}
 \item If $[\gamma] = [x_1,\dots,x_n \leftarrow t]$, then $t(\gamma) = t$  and $[\gamma]^v = \{\sn(t)^1/x_1,\dots,\sn(t)^n/x_n\}[\sh (t)]$
 \item If $[\gamma] = [x_1,\dots,x_n \twoheadleftarrow \lambda y\langle t^1,\dots,t^n \rangle[\Gamma]]$ with $t^1,\dots,t^n$  fresh variants of $t$, then  $t(\gamma) = \lambda y.t[\Gamma]$  and $[\gamma]^v = u\{\sn(\lambda y.t[\Gamma])^1/x_1,\dots,\sn(\lambda y.t[\Gamma])^n/x_n\}[\sh (\lambda y.t[\Gamma])]$.
\end{itemize}

\begin{ALlemma}\label{lem:CompSubst}
 $[\gamma]^v \{w/x\} \rightsquigarrow [\gamma\{w/x\} ]^v$
\end{ALlemma}

\begin{proof}
Immediate by definition.
\end{proof}

\begin{ALlemma}\label{lem:IntCaseSharing0}
For $[\Gamma] = [\gamma_1] \dots[\gamma_p]$, we denote by $[\Gamma]^*$ the sequence $[\gamma_1]^* \dots[\gamma_p]^*$, where $[\gamma_i]^*$ is either $[\gamma_i]$ or $[\gamma_i]^v$.
 If $(u[\Gamma_0]^*)\overline{w[\Gamma]^*} \in \SN$ and for each closure $[\gamma]$ of $[\Gamma_0]\overline{[\Gamma]}$,  $\term (\gamma)\in\SN$, then $(u[\Gamma])\overline{w[\Gamma]} \in \SN$.
\end{ALlemma}

\begin{proof}

  Let $T = (u[\Gamma_0])\overline{w[\Gamma]}$ and $T' = (u[\Gamma_0]^*)\overline{w[\Gamma]^*}$.
\\
We proceed by induction on $Mes(T,T') = (S(T'), m_1(T), m_2(T))$ where

 $m_1(T)$ is the sum of the $S(t(\gamma))$ for  $[\gamma]$ in $[\Gamma_0]\overline{[\Gamma]}$

 $m_2(T)$ is the sum of the $lh(t(\gamma))$ for $[\gamma]$ in $[\Gamma_0]\overline{[\Gamma]}$
 \\
If,  for each closure $[\gamma]$ of $[\Gamma_0]\overline{[\Gamma]}$,  $[\gamma]^* = [\gamma]$, then the result is trivial (note that if $t(\gamma)$ is a variable, then  $[\gamma]^* = [\gamma]$). We suppose in the following that $[\gamma]^* \neq [\gamma]$ for at least one$[\gamma]$ of $[\Gamma_0]\overline{[\Gamma]}$ . We can also suppose that $u$ doesn't end with a losure, because it can be integrated in the sequence $[\Gamma_0]$.
\\
In order to show that $T\in\SN$, we show that each one-step reduction from $T$ gives a term $U\in\SN$. We consider the different possible reductions leading to $U$. If the reductions are inside $u$ or $\overline{w}$ we conclude as in Lemma~\ref{lem:IntCaseLambda0}. If it is the lifting of a closure $[\gamma]$ of $[\Gamma_0]\overline{[\Gamma]}$ using rule (2) we conclude also as in Lemma~\ref{lem:IntCaseLambda0}. It remains to consider the different possible reductions in a closure $[\gamma]$ of $[\Gamma_0]\overline{[\Gamma]}$ or the interaction of two such closures using rule (9). For simplication we consider the case where the reduction involves $[\gamma_1]$.
%
%
\medskip
\\
1) Suppose that $U$ is obtained by the reduction a closure $[\gamma]$ such that  $[\gamma]^* = [\gamma]$. Suppose for instance that is closure $[\gamma_i]$ of $[\Gamma_0] = [\gamma_1] \dots[\gamma_p]$. This reduction  may apply substitutions to $u$ and $\gamma_j$, for $j<i$, transforming them into $u^S$ and $\gamma_j^S$. We have in this case $U = (u^S[\gamma_1]^S \dots [\gamma'_i]\dots [\gamma_p])\overline{w[\Gamma]}$.
\\
Consider $U' = (u^S[\gamma_1]^{S*} \dots [\gamma'_i]\dots [\gamma_p]^*)\overline{w[\Gamma]^*}$ (note that that by $[\gamma_1]^{S*}$ we mean that we apply first the substitution and then computation, and by $[\gamma_1]^{*S}$, the converse).  We have

$
\begin{array}{llll}
 T' & = & (u[\gamma_1]^* \dots[\gamma_p]^*)\overline{w[\Gamma]^*} & \\
    & \rightsquigarrow^1 & (u^S[\gamma_1]^{*S} \dots [\gamma'_i]\dots [\gamma_p]^*)\overline{w[\Gamma]^*} & \\
    & \rightsquigarrow & (u^S[\gamma_1]^{S*} \dots [\gamma'_i]\dots [\gamma_p]^*)\overline{w[\Gamma]^*} & \hspace{0.5cm} \mbox{by Lemma~\ref{lem:CompSubst}  }\\
    & = & U' &
\end{array}
$
\\
Therefore $U'\in\SN$ and $S(U')<S(T')$. Because $Mes(U,U') < Mes(T,T')$, we have by induction hypothesis, $U\in\SN$.
%
%
\medskip
\\
2) Suppose that $U$ is obtained by the reduction a closure $[\gamma]$ such that  $[\gamma]^* = [\gamma]^v$. For simplifying we suppose that it is the first closure of $[\Gamma_0]$, i.e. it is $[\gamma_1]$ and $[\Gamma_0] = [\gamma_1][\Gamma'_0]$.
\smallskip
\\
2.1)  $T = (u[x_1,\dots,x_n \leftarrow t][\Gamma'_0])\overline{w[\Gamma]}$ and $U = (u[x_1,\dots,x_n \leftarrow t'][\Gamma'_0])\overline{w[\Gamma]}$ with $t\rightsquigarrow^1 t'$.
\\
In this case $T' = (u\{\sn(t)^1/x_1,\dots,\sn(t)^n/x_n\}[\sh (t)][\Gamma'_0]^*)\overline{w[\Gamma]^*}$.
\\
Consider $U' = (u\{\sn(t')^1/x_1,\dots,\sn(t')^n/x_n\}[\sh (t')][\Gamma'_0]^*)\overline{w[\Gamma]^*}$. We have $\sh (t') = \sh (t)$ and by Lemma~\ref{lem:lift-sn}, $\sn (t) \rightsquigarrow \sn (t')$. Therefore $T'\rightsquigarrow U'$ and $U'\in\SN$. We have also $S(U')\leq S(T')$ and $S(t')<S(t)$. It follows that  $m_1(U) < m_1(T)$ and  $Mes(U,U') < Mes(T,T')$ and we have by induction hypothesis, $U\in\SN$.
%
\smallskip
\\
2.2)
$T =  (u\distr xyt[\Gamma_0'])\overline{w\G}$ and
$U =  (u\distr xy{t'}[\Gamma_0'])\overline{w\G}$, with $t ~\rewrite^1~ t'$.
In this case
$T' = (u\sub{x_1}{\sn(\lambda y.\pi_1(t))^1}\dots\sub{x_n}{\sn(\lambda
y.\pi_n(t))^n}[\sh(\lambda y.t)][\Gamma_0']^*)\overline{w\G^*}$.
Consider
$U' = (u\sub{x_1}{\sn(\lambda
y.\pi_1(t'))^1}\dots\sub{x_n}{\sn(\lambda y.\pi_n(t'))^n}[\sh(\lambda
y.t')][\Gamma_0']^*)\overline{w\G^*}$.
We have $[\sh(\lambda y.t)]=[\sh(\lambda y.t')]$ and by lemma\ref{lem:lift-sn},
$\sn(\lambda y.\pi_1(t)) \rewrite^\star \sn(\lambda y.\pi_1(t'))$.
Therefore $T'\rewrite U'$ and $U'\in\SN$. We have also that
$S(U')\leq S(T')$ and $S(t') < S(t)$. It follows that $m_1(U)<m_1(T)$
and $Mes(U,U') < Mes(T,T')$ and we have by induction hypothesis,
$U\in\SN$.
%
% 2.2)  $T = (u[x_1,\dots,x_n \leftarrow (t)v][\Gamma'_0])\overline{w[\Gamma]}$ and
% \\
% $U = (u\{(\alpha_1)\beta_1 /x_1,\dots, (\alpha_n)\beta_n/x_n\}[\alpha_1,\dots,\alpha_n \leftarrow t][\beta_1,\dots,\beta_n \leftarrow v][\Gamma'_0])\overline{w[\Gamma]}$.
% \\
% In this case $T' = (u\{\sn((t)v)^1/x_1,\dots,\sn((t)v)^n/x_n\}[\sh ((t)v)][\Gamma'_0]^*)\overline{w[\Gamma]^*}$.
% \\
% Consider $U' = (u\{(\alpha_1)\beta_1 /x_1,\dots, (\alpha_n)\beta_n/x_n\}
% \{\sn(t)^1/\alpha_1,\dots,\sn(t)^n/\alpha_n\}[\sh(t)]\{\sn(v)^1/\beta_1,\dots,\sn(v)^n/\beta_n \leftarrow v\}[\sh(v)][\Gamma'_0]^*)\overline{w[\Gamma]^*}$.
% \\
% We have
% \begin{eqnarray*}
%  U' & = & (u \{(\sn (t)^1)\sn (v)^1 /x_1,\dots, (\sn (t)^n)\sn(t)^n/x_n\}[\sh(t)][\sh(v)][\Gamma'_0]^*)\overline{w[\Gamma]^*}\\
%     & = & (u\{\sn((t)v)^1/x_1,\dots,\sn((t)v)^n/x_n\}[\sh ((t)v)][\Gamma'_0]^*)\overline{w[\Gamma]^*} \\
%     & = & T'
% \end{eqnarray*}
% Therefore $U'\in\SN$ and $S(U') = S(T')$. We also have  $m_1(U) \leq m_1(T)$ and because $lh(t) + lh(v) < lh((t)v)$, $m_2(U) < m_2(T)$ and  $Mes(U,U') < Mes(T,T')$. Thus by induction hypothesis, $U\in\SN$.
%
\smallskip
\\
2.3)  $T = (u[x_1,\dots,x_n \leftarrow (t)v][\Gamma'_0])\overline{w[\Gamma]}$ and
\\
$U = (u\{(\alpha_1)\beta_1 /x_1,\dots, (\alpha_n)\beta_n/x_n\}[\alpha_1,\dots,\alpha_n \leftarrow t][\beta_1,\dots,\beta_n \leftarrow v][\Gamma'_0])\overline{w[\Gamma]}$.
\\
In this case $T' = (u\{\sn((t)v)^1/x_1,\dots,\sn((t)v)^n/x_n\}[\sh ((t)v)][\Gamma'_0]^*)\overline{w[\Gamma]^*}$.
\\
Consider $U' = (u\{(\alpha_1)\beta_1 /x_1,\dots, (\alpha_n)\beta_n/x_n\}
\{\sn(t)^1/\alpha_1,\dots,\sn(t)^n/\alpha_n\}[\sh(t)] \\
\hspace*{62pt}\{\sn(v)^1/\beta_1,\dots,\sn(v)^n/\beta_n \leftarrow v\}[\sh(v)][\Gamma'_0]^*)\overline{w[\Gamma]^*}$.
\\
We have
\begin{eqnarray*}
 U' & = & (u \{(\sn (t)^1)\sn (v)^1 /x_1,\dots, (\sn (t)^n)\sn(t)^n/x_n\}[\sh(t)][\sh(v)][\Gamma'_0]^*)\overline{w[\Gamma]^*}\\
    & = & (u\{\sn((t)v)^1/x_1,\dots,\sn((t)v)^n/x_n\}[\sh ((t)v)][\Gamma'_0]^*)\overline{w[\Gamma]^*} \\
    & = & T'
\end{eqnarray*}
Therefore $U'\in\SN$ and $S(U') = S(T')$. We also have  $m_1(U) \leq m_1(T)$ and because $lh(t) + lh(v) < lh((t)v)$, $m_2(U) < m_2(T)$ and  $Mes(U,U') < Mes(T,T')$. Thus by induction hypothesis, $U\in\SN$.
%
\smallskip
\\
2.4)  $T = (u[x_1,\dots,x_n \leftarrow \lambda y.t][\Gamma'_0])\overline{w[\Gamma]}$ and
\\
$U = (u[x_1,\dots,x_n \twoheadleftarrow \lambda y. \langle \alpha_1,\dots,\alpha_n \rangle [\alpha_1,\dots,\alpha_n \leftarrow  t]][\Gamma'_0])\overline{w[\Gamma]}$.
\\
In this case $T' = (u\{\sn(\lambda y.t)^1/x_1,\dots,\sn(\lambda y.t)^n/x_n\}[\sh (\lambda y.t)][\Gamma'_0]^*)\overline{w[\Gamma]^*}$.
\\
Consider $U' = (u\{\sn (\lambda y.\alpha[\alpha \leftarrow t])/x_1,\dots, \sn (\lambda y.\alpha[\alpha \leftarrow t])/x_n\}[\sh (\lambda y.\alpha[\alpha \leftarrow t])][\Gamma'_0]^*)\overline{w[\Gamma]^*}$.
\\
We have  $U' = (u\{\sn(\lambda y.t)^1/x_1,\dots,\sn(\lambda y.t)^n/x_n\}[\sh (\lambda y.t)][\Gamma'_0]^*)\overline{w[\Gamma]^*} = T'$
%\EMPTY{(check this equality)}
\\
Therefore $U'\in\SN$ and $S(U') = S(T')$. We also have  $m_1(U) \leq m_1(T)$ and because \\
$lh(\lambda y. \langle \alpha_1,\dots,\alpha_n \rangle [\alpha_1,\dots,\alpha_n \leftarrow  t]) = lh(t) < lh(\lambda y.t)$, $m_2(U) < m_2(T)$ and  $Mes(U,U') < Mes(T,T')$. Thus by induction hypothesis, $U\in\SN$.
%
\smallskip
\\
2.5)  $T = (u[x_1,\dots,x_n \twoheadleftarrow \lambda y. \langle t^1,\dots,t^n \rangle [\overline{z^1},\dots,\overline{z^n}\leftarrow  y]][\Gamma'_0])\overline{w[\Gamma]}$ and
\\
$U = (u\{  \lambda y^1.t^1[\overline{z^1}\leftarrow y^1]/x_1, \dots, \lambda y^n.t^n[\overline{z^n}\leftarrow y^n]/x_n \}[\sh (\lambda y.t[\overline{z}\leftarrow y])][\Gamma'_0])\overline{w[\Gamma]}$.
\\
In this case $T' = ( u\{\sn(\lambda y.t[\overline{z}\leftarrow  y])^1/x_1,\dots,\sn(\lambda y.t[\overline{z}\leftarrow  y])^n/x_n\}[\sh (\lambda y.t[\overline{z}\leftarrow  y])][\Gamma'_0]^*)\overline{w[\Gamma]^*}$.
\\
Consider $U' = (u\{  \lambda y^1.t^1[\overline{z^1}\leftarrow y^1]/x_1, \dots, \lambda y^n.t^n[\overline{z^n}\leftarrow y^n]/x_n \}[\sh (\lambda y.t[\overline{z}\leftarrow y])][\Gamma'_0]^*)\overline{w[\Gamma]^*}$.
%
Because $t^i$ is already in unfolding, we have  $\sn(\lambda y.t)^i = \sn(\lambda y.t[\overline{z}\leftarrow  y])^i$.
%(\EMPTY{Here we use not only the fact that the $t^i$ are variants with the convention on variables, but also that they are in unfolding, i.e. produced by reduction)}
It follows $T' = U'$, $U'\in\SN$ and $S(U') = S(T')$.
\\
Therefore $U'\in\SN$ and $S(U') = S(T')$. We  have  $m_1(U) < m_1(T)$ and therefore  $Mes(U,U') < Mes(T,T')$. Thus by induction hypothesis, $U\in\SN$.
%
%
%
%
\medskip
\\
3)  Suppose that U is obtained by the interaction of the two first closures of $[\Gamma_0]$ using rule (9). It is enough to consider only the non trivial case where the term of the second closure is not a variable. We have \\
$T = (u[y_1,\dots,y_m  \leftarrow x_1][x_1,x_2\dots,x_n \leftarrow t][\Gamma'_0])\overline{w[\Gamma]}$   and
\\
$U = (u[y_1,\dots,y_m,x_2\dots,x_n \leftarrow t][\Gamma'_0])\overline{w[\Gamma]}$.
\\
Let $z_1,\dots,z_p$ the free variables of $t$. By our convention, theses variables are renamed $z_1^i,\dots,z_p^i$ in $\sn(t)^i$ and these are exactly the free variables of  $\sn(t)^i$. In the following we will have to consider $\sn (\sn (t)^1)^j$. Because $\sn(t)$ is in unfoldingal form, $\sn(\sn(t)) = \sn(t)$ and $\sn (\sn (t)^1)^j$ is obtained from $\sn(t)$ by replacing any variable $x$ by $(x^1)^j$: we will  adopt the notation $x^{1,j}$ for $(x^1)^j$ and $\sn(t)^{1,j}$ for $\sn (\sn (t)^1)^j$.
%
We have
%
\begin{eqnarray*}
 T' & = & (u[y_1,\dots,y_m \leftarrow x_1]\{\sn(t)^1/x_1,\dots,\sn(t)^n/x_n\}[\sh (t:1,\dots,n)][\Gamma]^*)\overline{w[\Gamma]^*} \\
    & = & (u[y_1,\dots,y_m \leftarrow \sn(t)^1]\{\sn(t)^2/x_2,\dots,\sn(t)^n/x_n\}[\sh (t:1,\dots,n)][\Gamma]^*)\overline{w[\Gamma]^*} \\
    & \rightsquigarrow & (u_1\{\sn(t)^{1,1}/y_1,\dots,\sn(t)^{1,m}/y_m\}[\sh (t^1:(1,1),\dots,(1,m))] \\
    &   & \{\sn(t)^2/x_2,\dots,\sn(t)^n/x_n\}[\sh (t:1,\dots,n)][\Gamma]^*)\overline{w[\Gamma]^*} \\
    & = & (u\{\sn(t)^{1,1}/y_1,\dots,\sn(t)^{1,m}/y_m\}\{\sn(t)^2/x_2,\dots,\sn(t)^n/x_n\} \\
    &   & [\sh (t^1:(1,1),\dots,(1,m))][\sh (t:1,\dots,n)][\Gamma]^*)\overline{w[\Gamma]^*}   \\
    & = & (u\{\sn(t)^{1,1}/y_1,\dots,\sn(t)^{1,m}/y_m, \sn(t)^2/x_2,\dots,\sn(t)^n/x_n\} \\
    &   & [\sh (t^1:(1,1),\dots,(1,m))][\sh (t:1,\dots,n)][\Gamma]^*)\overline{w[\Gamma]^*}
\end{eqnarray*}
%
Let $u_1 = u\{t^{1,1}/y_1,\dots,t^{1,m}/y_m\}\{t^2/x_2,\dots,t^n/x_n\}$ and $z_1,\dots,z_p$ the fre variables of $t$. We have
%
\begin{eqnarray*}
  &                   & u_1[\sh (t^1:(1,1),\dots,(1,m))][\sh (t:1,\dots,n)] \\
  & =                 & u_1[z_1^{1,1}, \dots, z_1^{1,m} \leftarrow z_1^1]\dots, [z_p^{1,1}, \dots, z_p^{1,m} \leftarrow z_p^1][z_1^1,\dots,z_1^n\leftarrow z_1]\dots [z_p^1,\dots,z_p^n\leftarrow z_1] \\
  & \rightsquigarrow  & u_1[z_1^{1,1}, \dots, z_1^{1,m}, z_1^2,\dots,z_1^n\leftarrow z_1]\dots [z_p^{1,1}, \dots, z_p^{1,m}, z_p^2,\dots,z_p^n\leftarrow z_p] \\
  &  =                & u_1[\sh (t:(1,1),\dots,(1,m),2,\dots,n)]
\end{eqnarray*}
Consider $U' = (u\{\sn(t)^{1,1}/y_1,\dots,\sn(t)^{1,m}/y_m, \sn(t)^2/x_2,\dots,\sn(t)^n/x_n\} \\
\hspace*{62pt}[\sh (t:(1,1),\dots,(1,m),2,\dots,n)][\Gamma]^*)\overline{w[\Gamma]^*}$.
\\
We have $T'\rightsquigarrow U'$ and therefore $U'\in\SN$; because $t$ is not a variable, $S(U')<S(T')$.
%\TODO{check in the limit cases (eg nullary sharings) that there is always one step of reduction}
Because $Mes(U,U') < Mes(T,T')$, we have by induction hypothesis, $U\in\SN$.


\end{proof}




% =================================================================================================

\section{Intersection Types}

\begin{ALtheorem}\label{thm:SND}
If $x_1:A_1,\dots, x_n:A_n \vdash_D u:B$ then $u \in \SN$.
\end{ALtheorem}

\noindent
As a consequence of this theorem we get the preservation of the strong normalisation with respect to lambda-calculus (often called PSN property), using the well know fact that the strongly normalisable lambda-terms are typable in $D$ (\cite{Coppo-DezaniCiancaglini-1980,Pottinger-1980,Krivine-1993}).

\begin{ALtheorem}[PSN]
If $N$ is strongly normalisable then $\tercoden N$ is strongly normalisable.
\end{ALtheorem}

\begin{proof}
If $N$ is strongly normalisable, then it is typeable in $D$; then by a generalization of Proposition~\ref{prop:types preserved}, $\tercoden N$ is typeable in $D_a$, and by Theorem~\ref{thm:SND}, $\tercoden N$ is strongly normalisable.
\end{proof}

\section{Conclusions and further work}



The present result, of strong normalisation for the atomic lambda-calculus with intersection types, emphasises how the calculus is a natural and well-behaved formalisation of sharing in the lambda-calculus.
%
Future investigations will expand in three directions: strengthening the current strong normalisation result; adapting the atomic lambda-calculus to address further notions of sharing; and investigating the practical use of the calculus in computation, for instance in compiling or implementing functional programming languages.



The present work suggests two immediate angles for future research.
%
Firstly, to show that intersection types characterise exactly the strongly normalisable terms of the atomic lambda-calculus, it must be shown that any strongly normalisable term is typeable in $D_a$.
%
It is anticipated that this result will be proved soon.
%
Secondly, in the medium term, it is expected that the type system and strong normalisation proof can be extended to the second-order case---although subject reduction is not immediately obvious in this case.
%, where type inference (in e.g.\ a  Hindley--Milner type system) becomes useful


For the atomic lambda-calculus in general, further work will focus on variations on the calculus that more closely approach the reduction dynamics of sharing graphs, to encompass further degrees of sharing.
%
Another direction would be the inclusion of general recursion in the calculus, and the investigation of its interaction with the sharing constructs, as a prerequisite of making the calculus useful in practice to the implementation of functional programming languages.



\bibliographystyle{plain}
\bibliography{AL-LPAR}

\newpage

\section*{Appendix}



Proof of Lemma~\ref{lem:unsharing}:

\begin{proof}
The case $u\share x{\sn(t)}$ follows by structural induction on $\sn(t)$.
%
This gives the third step for the general case, in which
$u\share xt$ reduces as follows: $u\share xt$
\\ $ \rewrite u\share x{\sn(t)\share*{x_{1,1},\dots,x_{1,m_1}}{x_1}\dots\share*{x_{n,1},\dots,x_{n,m_n}}{x_n}}$
\\ $ \rewrite u\share x{\sn(t)}\share*{y_{1,1},\dots,y_{1,m_1}}{y_1}\dots\share*{y_{n,1},\dots,y_{n,m_n}}{y_n}$
\\ $ \rewrite u\sub{x_1}{\sn(t)^1}\dots\sub{x_n}{\sn(t)^n}[\sh(\sn(t))]\share*{y_{1,1},\dots,y_{1,m_1}}{y_1}\dots\share*{y_{n,1},\dots,y_{n,m_n}}{y_n}$
\\ $\rewrite u\sub{x_1}{\sn(t)^1}\dots\sub{x_n}{\sn(t)^n}[\sh(t)]$
\end{proof}

Proof of Lemma~\ref{lem:undist}:

\begin{proof}
Analogous to the proof of Lemma~\ref{lem:unsharing}.
\end{proof}

Proof of Lemma~\ref{lem:lift-sn}:

\begin{proof}
The following three cases are proven simultaneously.
\\ 1) If $u ~\ALnotbeta~ u'$, then $\sn(u) ~=~ \sn(u')$.
\\ 2) If $u ~\ALbeta~ u'$, then the statement follows by structural induction on $u$.
\\ 3) If $t=\tup t\G ~\rewrite~ t'=\tup{t'}\G*$, by case 2 (and Proposition~\ref{prop:sn_pi}) the result follows from the reduction
%
	$t_i\share*{}{x_1}\dots\share*{}{x_m}\G\rewrite
	 t'_i\share*{}{x_1}\dots\share*{}{x_m}\G*$,
which will be shown next.

If the reduction happens inside $\vv t$, or inside $\G$ without substituting into
$\vv t$, the result is immediate.
%
Otherwise, write $t'=\tup t\sub{y_{1,1}}{u_{1,1}}\sub{y_{m,k_m}}{u_{m,k_m}}\G*$, where $\FV(t_i)=\{y_{i,1},...,y_{i,k_i}\}$.
%
Without loss of generality, consider $i=1$. Then
%
\[
\begin{aligned}
\pi_1(t)
	&= t_1\share*{}{y_{1,1}}\dots\share*{}{y_{m,k_m}}\G
\\ &\rewrite t_1\sub{y_{1,1}}{u_{1,1}}\sub{y_{1,k_1}}{u_{1,k_1}}\share*{}{u_{2,1}}\dots\share*{}{u_{m,k_m}}\G*
\\ &\rewrite t_1\sub{y_{1,1}}{u_{1,1}}\sub{y_{1,k_1}}{u_{1,k_1}}\share*{}{z_1}\dots\share*{}{z_l}\G*
\\ &= \pi_1(t')\;,
\end{aligned}
\]
%
where $\{z_1, \dots, z_l\}$ are the free variables of $u_{2,1}$,
$\dots$, $u_{m,k_m}$.

\end{proof}

\begin{ALlemma}
\label{lem:lift-sn}
For any terms $u$ and $u'$ if $u\rewrite u'$, then $\sn(u)
~\rewrite^\star~ \sn(u')$. Moreover, for any terms of multiplicity
$n$, $t$ and $t'$, and any $i\leq n$, if $t\rewrite t'$, then
$\sn(\pi_i(t))\rewrite^\star\sn(\pi_i(t'))$.
\end{ALlemma}


Proof of Lemma~\ref{lem:HeadVar}:

\begin{proof}
Every reduction step in $(x)\overline{t}$ can be lifted to a reduction
step in one of the terms of $\overline{t}$, with the exception of
reductions on the form $(u\g)v\ALnotbeta(u)v\g$. The crucial fact is
that if $(u\g)v\ALnotbeta(u)v\g$, then a reduction step in $(u)v\g$
can be lifted to either $u\g$ or $v$, as $\g$ can not capture any
variables in $v$, and hence can not interact with it.
%
As there may only be finitely many reduction steps of a form that
cannot be lifted, the result follows by contradiction.
\end{proof}

Proof of Lemma~\ref{prop:IntSubst}

\begin{proof}
We prove by induction on $A$ that (i) $ |A| \subseteq  \SN$ and (ii) for each $x\in \VAR$ and $\overline{t} \in \SN^{<\omega}$, $(x)\overline{t} \in |A|$.
\\
1) If $A=X$, then $|A| =\SN$ and the result follows from Lemma~\ref{lem:HeadVar}.
\\
2) Suppose $A=C\vlim D$.
\\
(i) Let $u \in |A|$. For $x \in \VAR$, we have $x\in |C|$ by induction hypothesis; therefore $(u)x \in |D|$ and by induction hypothesis $(u)x \in \SN$. It follows  $u \in \SN$.
\\
(ii) Let $x\in \VAR$ and $\overline{t} \in \SN^{<\omega}$. Let $v\in |C|$; then by induction hypothesis,  $v\in \SN$ and therefore, also by induction hypothesis, $((x)\overline{t})v \in |D|$. It follows that $(x)\overline{t} \in |C \vlim D|$.
\\
3) Suppose $A=C \cap D$. By definition of $|-|$, we have $u\in |A|$ iff $u\in |C|$ and $u\in |D|$, and the result follows directly by induction hypothesis.
\end{proof}

\begin{ALlemma}\label{lem:SN-AddSharings}
If $(u)\overline{w} \in \SN$, then $(u\share xx)\overline{w} \in \SN$.
\end{ALlemma}


\begin{proof}
First show the result for empty $\overline{w}$, by contradiction.
%
Let $w_0\rewrite^1 w_1\rewrite^1 \ldots$ be an infinite reduction from $w_0=u\share xx$.
%
Each $w_i$ can be written as $w'_i\G[i]$, where $\G[i]$ the largest possible sequence of sharings of the form $\share xx$ (with the body a variable).
%
A rewrite $w_i\rewrite^1 w_{i+1}$ is of one of the following three forms: \textit{a)} it is inside $w'_i$, \textit{b)} it is inside $\G[i]$, or \textit{c)} it rewrites $w'_i$ to $w'_{i+1}\G*$, so that $\G[i+1]=\G*\G[i]$.
\[
		a)\quad	w'_i\G[i] \rewrite^1 w'_{i+1}\G[i]
\qquad	b)\quad	w'_i\G[i] \rewrite^1 w'_i\G[i+1]
\qquad	c)\quad	w'_i\G[i] \rewrite^1 w'_{i+1}\G*\G[i]
\]
The reduction path from $w_0$ is modified as follows: steps of type \textit{a)} and \textit{c)} are retained (up to renaming of variables), but those of type \textit{b)} are omitted.
%
It will be shown that the resulting path \textit{1)} is well-defined, \textit{2)} is infinite, and \textit{3)} generates an infinite reduction from $u$.


First, note that a rewrite step of type \textit{b)} must be an instance of rule \eqref{eqn:execute unary substitution} or \eqref{eqn:compound sharings}.
%
In the latter case, $w'_i$ remains unaffected, i.e.\ $w'_i=w'_{i+1}$; in the former case, a sharing $\share xy$ is evaluated by a substitution $\sub xy$; then $w'_i$ and $w'_{i+1}$ are identical up to renaming of $x$ by $y$.
%
This proves \textit{1)}.
%
For \textit{2)}, each step of type \textit{b)} reduces the length of $\G[i]$ by 1; thus, if the original infinite reduction from $w_0$ contains infinitely many steps of type \textit{b)}, it must also contain infinitely many of type \textit{c)}.
%
For \textit{3)} it suffices to observe that in the new path no rewrite step affects the sharing $\share xx$, since the latter is part of $\G[0]$; thus, the new reduction path from $w_0$ immediately gives and infinite reduction from $u$.

Lastly, the full result follows as $x$ must be free in $(u\share
xx)\overline{w}$, and hence all but a finite number of reduction steps
may be lifted from $(u\share xx)\overline{w}$ to
$(u)\overline{w}\share xx$.

%\TODO{ write a condensed proof}
%
\end{proof}




Proof of Lemma~\ref{lem:Red-AddSharings}

\begin{proof}
 We proceed by induction on $B$.
\\
1) $B$ is a variable. Then $|B| = \SN$ and the result is given by Lemma~\ref{lem:SN-AddSharings}.
\\
2) $B= C\vlim D$.
Suppose $(u)\overline{w} \in |C\vlim D|$. Let $t\in|C|$. We prove that  $((u\share xx)\overline{w})t \in |D|$. Because $(u)\overline{w} \in |C\vlim D|$ and $t\in|C|$, we have $((u)\overline{w})t \in |D|$ and by induction hypothesis, $((u\share xx)\overline{w})t \in |D|$. It follows that $(u\share xx)\overline{w} \in |C\vlim D|$.
\\
3) $B= C\cap B$.
The result directly follows from the induction hypothesis using the fact that $|C\cap D| = |C|\cap|D|$.

\end{proof}


Proof of Lemma~\ref{lem:IntCaseLambda0}:

\begin{proof}
 Let $T = ((((\lambda x.u)[\Gamma]) v)[\Gamma'])\overline{w[\Gamma]}$ and $T' = ((u[v/x])[\Gamma][\Gamma'])\overline{w[\Gamma]}$.
\\
In order to show that $T\in\SN$, we show that each one-step reduction from $T$ gives a term $U\in\SN$.
%
We proceed by induction on $Mes(T) = (S(T'), lh(u), lh(v)), lh([\gamma])$ where
 $lh([\Gamma])$ is the length of the sequence $[\Gamma]$.
\\
We consider the different possibilities for $U$.
\smallskip
\\
1)  $U = ((u\{v/x\})[\Gamma][\Gamma'])\overline{w[\Gamma]}$ (in this case $[\Gamma]$ is empty).
\\
We have $U=T'\in\SN$.
\smallskip
\\
2)  $U = ((((\lambda x.u^*)[\Gamma]) v)[\Gamma'])\overline{w[\Gamma]}$ with $u\rightsquigarrow^1 u^*$.
\\
Consider $U' = ((u^*\{v/x\})[\Gamma][\Gamma'])\overline{w[\Gamma]}$. We have $T'\rightsquigarrow^1 U'$ and therefore $U'\in\SN$ and $S(U')<S(T')$. Because $Mes(U) < Mes(T)$, we have by induction hypothesis, $U\in\SN$.
\smallskip
\\
3)  $U = ((((\lambda x.u_1)[\gamma][\Gamma]) v)[\Gamma'])\overline{w[\Gamma]}$ with $u = u_1[\gamma]$ and $x \notin FV(\gamma)$.
\\
Consider $U' = ((u_1\{v/x\})[\gamma][\Gamma][\Gamma'])\overline{w[\Gamma]}$.
We have
\\
$T' = (((u_1[\gamma])\{v/x\})[\Gamma][\Gamma'])\overline{w[\Gamma]} = (((u_1\{v/x\})[\gamma])[\Gamma][\Gamma'])\overline{w[\Gamma]} = U'$ and $U'\in\SN$.
Because $S(U') = S(T')$ and $lh(u_1) < lh(u)$, we have $Mes(U) < Mes(T)$ and by induction hypothesis, $U\in\SN$.
\smallskip
\\
4)  $U = ((((\lambda x.u)[\Gamma^*]) v)[\Gamma'])\overline{w[\Gamma]}$ with $[\Gamma^*]$ is obtained by a one-step reduction in $[\Gamma]$ using any rule except \eqref{eqn:execute unary substitution}, \eqref{eqn:share application}, \eqref{eqn:distributor elimination}.
\\
Consider $U' = ((u\{v/x\})[\Gamma^*][\Gamma'])\overline{w[\Gamma]}$.  We have $T'\rightsquigarrow^1 U'$ and therefore $U'\in\SN$ and $S(U')<S(T')$. Because $Mes(U) < Mes(T)$, we have by induction hypothesis, $U\in\SN$.
\\
The same argument holds if $U$ is obtained from $T$ by any one-step reduction in $[\Gamma']\overline{w[\Gamma]}$ using any rule except \eqref{eqn:execute unary substitution}, \eqref{eqn:share application}, \eqref{eqn:distributor elimination}.
\smallskip
\\
5)  $U = ((((\lambda x.u^*)[\Gamma^*]) v)[\Gamma'])\overline{w[\Gamma]}$ and $U$ is obtained from $T$ by a one-step reduction in $[\Gamma]$ using a rule among \eqref{eqn:execute unary substitution}, \eqref{eqn:share application}, \eqref{eqn:distributor elimination} (these rules may apply substitutions to $u$ transforming it into $u^*$).
\\
Consider $U' = ((u^*\{v/x\})[\Gamma^*][\Gamma'])\overline{w[\Gamma]}$. $U'$ is obtained from $T'$ by applying the same rule $[\Gamma]$ and therefore $U'\in\SN$ and $S(U')<S(T')$. Because $Mes(U) < Mes(T)$, we have by induction hypothesis, $U\in\SN$.
\\
The same argument holds if $U$ is obtained from $T$ by any one-step reduction in $[\Gamma']\overline{w[\Gamma]}$ using a rule among \eqref{eqn:execute unary substitution}, \eqref{eqn:share application}, \eqref{eqn:distributor elimination} (in this case the substitution may transform also $u$, $v$ and $[\Gamma]$).
\smallskip
\\
6)  $U = ((((\lambda x.u)[\Sigma]) v)[\gamma][\Gamma'])\overline{w[\Gamma]}$ with $[\Gamma] = [\Sigma][\gamma]$ (application of rule \eqref{eqn:sharing above application function})
\\
Consider $U' = ((u\{v/x\})[\Sigma][\gamma][\Gamma'])\overline{w[\Gamma]}$.
We have $T' = U'$ and therefore $U'\in\SN$ and $S(U') = S(T')$. Because $lh([\Sigma]) < lh([\Gamma])$, we have $Mes(U) < Mes(T)$ and by induction hypothesis, $U\in\SN$.
\smallskip
\\
7)  $U = ((((\lambda x.u)[\Gamma]) v_1)[\gamma][\Gamma'])\overline{w[\Gamma]}$ with $v = v_1[\gamma]$.
\\
Consider $U' = ((u\{v_1/x\})[\Gamma][\gamma][\Gamma'])\overline{w[\Gamma]}$.
We have $T' = (((u)\{v_1[\gamma]/x\})[\Gamma][\Gamma'])\overline{w[\Gamma]}$. $T'$ reduce to $U'$ in $n\ge 0$ steps (it can be $0$ in the case where $u = x$). Therefore $U'\in\SN$ and $S(U') \leq S(T')$. Because $lh(v_1) < lh(v)$, we have $Mes(U) < Mes(T)$ and by induction hypothesis, $U\in\SN$.
\smallskip
\\
8)  $U = ((((\lambda x.u)[\Gamma]) v*)[\Gamma'])\overline{w[\Gamma]}$ with $v\rightsquigarrow^1 v^*$.
\\
Consider $U' = ((u\{v^*/x\})[\Gamma][\Gamma'])\overline{w[\Gamma]}$. We have $T'\rightsquigarrow^1 U'$ (because $x$ has exactly one occurrence in $u$) and therefore $U'\in\SN$ and $S(U')<S(T')$. Because $Mes(U) < Mes(T)$, we have by induction hypothesis, $U\in\SN$.

\end{proof}


Proof of Lemma~\ref{lem:IntCaseLambdaSharing}:

\begin{proof}
 We proceed by induction on $B$.
\\
1) $B$ is a variable. Then $|B| = \SN$ and the result is given by Lemma~\ref{lem:IntCaseLambda0}.
\\
2) $B= C\vlim D$.
Suppose $(u\{v/x\})\overline{w} \in |C\vlim D|$ and $v\in\SN$. Let $t\in|C|$. We prove that  $(((\lambda x.u) v)\overline{w})t \in |D|$. Because $(u\{v/x\})\overline{w} \in |C\vlim D|$ and $t\in|C|$, we have $((u\{v/x\})\overline{w})t \in |D|$ and by induction hypothesis, $(((\lambda x.u) v)\overline{w})t \in |D|$. It follows that $((\lambda x.u) v)\overline{w} \in |C\vlim D|$.
%\\
%3) $B= C\cap B$.
%The result directly follows from the induction hypothesis using the fact that $|C\cap D| = |C|\cap|D|$.
\end{proof}
%
% \Comment{For proving Proposition~\ref{prop:IntSubst} (case 3), we need the statement of Lemma~\ref{lem:IntCaseLambda} only in the case $\overline{w}$ is empty. But for the inductive proof of this statement, the general statement of Lemma~\ref{lem:IntCaseLambda} is needed (case 2 of the proof). \\
%   Then for proving case 1 ($|B| = \SN$), we need the more complicated statement of Lemma~\ref{lem:IntCaseLambda0}: the proof by induction of Lemma~\ref{lem:IntCaseLambda0} introduces the case of sharing (case xx of the proof of Lemma~\ref{lem:IntCaseLambda0})  }
% %=================================
%
% \subsection{Preservation of strong normalisation}
%
%
% Figure~\ref{fig:lambda intersection types} presents a calculus of intersection types for the standard lambda-calculus.
% %
% The treatment of contexts $\Gamma$ is chosen to allow a close correspondence with derivations in $D_a$.
%
%
% \begin{ALdefinition}
% The type system $D$ of intersection types for the standard lambda-calculus $\Lambda$ is given by the rules in Figure~\ref{fig:lambda intersection types}.
% \end{ALdefinition}
%
%
% The type system characterises precisely the strongly normalisable terms.
%
%
% \begin{ALtheorem}[\cite{??}]\label{thm:lambda intersection SN}
% A lambda term $N\in\Lambda$ is SN if and only if $N$ is typeable in $D$.
% \end{ALtheorem}
%
% \begin{figure}[!tp]
% \[
%   \begin{array}{c@{\quads3}c@{\quads3}c}
% 	  \vlinf{}{\typevar}{x:A\vdash x:A}{}
% 	&
% 	  \vlinf{}{\typeabs}
% 	   {\Gamma \vdash \lambda x.N : A \vlim B}
% 	   {\Gamma, x : A,\dotsc, x : A \vdash N : B}
% 	&
% 	  \vliiinf{}{\typeapp}
% 	   {\Gamma,\Delta \vdash (N)M : B}
% 	   {\Gamma \vdash N : A\vlim B}
% 	   {}
% 	   {\Delta \vdash M : A}
% 	\\ \\ \\
%       \vlinf{}{\cap_{e1}}
% 	   {\Gamma \vdash N : A}
% 	   {\Gamma \vdash N : A\cap B}
% 	&
%       \vlinf{}{\cap_{e2}}
% 	   {\Gamma \vdash N : B}
% 	   {\Gamma \vdash N : A\cap B}
% 	&
% 	  \vliiinf{}{\cap_i}
% 	   {\Gamma \vdash N : A\cap B}
% 	   {\Gamma \vdash N : A}
% 	   {}
% 	   {\Gamma \vdash N : B}
%   \end{array}
% \]
% \caption{Type system $D$ for the lambda-calculus}
% \label{fig:lambda intersection types}
% \end{figure}
%
%
% \begin{ALproposition}\label{prop:intersection types preserved}
% If $N:A$ in $D$ then $\tercoden N:A$ in $D_a$.
% \end{ALproposition}
%
% \begin{proof}
% %By induction on the typing derivation for $\Gamma\vdash N:A$.
% %
% %The case where consists of a single axiom is immediate, and if the last rule in the derivation is one of $\typeapp$, $\cap_{e1}$, $\cap_{e2}$, or $\cap_i$, the induction hypothesis applies immediately.
% %
% A typing derivation for $\Gamma \vdash N : A$ translates directly to one for $\Gamma\vdash \tercoden N : A$, with the following notes:
% %
% 1) the derivation for $\tercoden N$ ends in an additional series of $(\typeshare)$-inferences as displayed below left; and
% %
% 2) a $(\lambda)$-inference in the derivation for $N$ translates to a consecutive $(\lambda)$-inference and $(\typeshare)$-inference, as illustrated below right.
% %
% \[
% 	{\vliiinf{}{\typeshare}
% 	   {\Gamma, x_i : B  \vdash u\share*{x^1_i,\dotsc,x^n_i}{x_i} : A}
% 	   {\Gamma, x^1_i \dots x^n_i : B \vdash u : A }
% 	   {}
% 	   {x_i : B \vdash x_i : B }
% 	}
% \qquad
% 	\vlderivation{
% 	 \vlin{}{\lambda}
% 	  {\Gamma \vdash \lambda x.u\share*{x^1,\dotsc,x^n}x : A\to B}
% 	  {\vliiin{}{\typeshare}
% 	   {\Gamma, x : B  \vdash u\share*{x^1,\dotsc,x^n}x : A}
% 	   {\vlhy{ \Gamma, x^1 \dots x^n : B \vdash u : A }}
% 	   {\vlhy{\ }}
% 	   {\vlhy{ x : B \vdash x : B }}
% 	  }
% 	}
% \quad
% \]
% \end{proof}
%
%
% \begin{ALtheorem}[PSN]
% If $N$ is SN then $\tercoden N$ is SN.
% \end{ALtheorem}
%
% \begin{proof}
% If $N$ is SN then by Theorem~\ref{thm:lambda intersection SN} it is typeable in $D$; then by Proposition~\ref{prop:intersection types preserved} $\tercoden N$ is typeable in $D_a$, and by Theorem~?? $\tercoden N$ is SN.
% \end{proof}

\end{document}
