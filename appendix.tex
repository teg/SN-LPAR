
% =================================================================================================
\section*{Appendix}

%\setcounter{ALtheorem}{0}
%\renewcommand\theALtheorem{\Alph{ALtheorem}}


In this appendix we complete the proof of strong normalisation for the simply-typed atomic lambda-calculus in Section~\ref{sec:SNproof}.
%
The postponed proofs of the lemmas and propositions of that section are treated in turn, with the statement of the lemma or proposition repeated.
%
Two additional lemmata are introduced as well.

\medskip
\noindent
\textbf{Lemma~\ref{lem:unsharing}.}
\quad
$u\share xt \rewrite u\subn{x_i}{\sn(t)^i}[\sh (t):1,\dots,n]$


\begin{proof}
The case $u\share x{\sn(t)}$ follows by structural induction on $\sn(t)$.
%
This gives the third step for the general case, in which
$u\share xt$ reduces as follows: $u\share xt$
\\ $ \rewrite u\share x{\sn(t)\share*{x_{1,1},\dots,x_{1,m_1}}{x_1}\dots\share*{x_{n,1},\dots,x_{n,m_n}}{x_n}}$
\\ $ \rewrite u\share x{\sn(t)}\share*{y_{1,1},\dots,y_{1,m_1}}{y_1}\dots\share*{y_{n,1},\dots,y_{n,m_n}}{y_n}$
\\ $ \rewrite u\sub{x_1}{\sn(t)^1}\dots\sub{x_n}{\sn(t)^n}[\sh(\sn(t))]\share*{y_{1,1},\dots,y_{1,m_1}}{y_1}\dots\share*{y_{n,1},\dots,y_{n,m_n}}{y_n}$
\\ $\rewrite u\sub{x_1}{\sn(t)^1}\dots\sub{x_n}{\sn(t)^n}[\sh(t)]$
\end{proof}


\medskip
\noindent
\textbf{Lemma~\ref{lem:undist}.}
$u\distr xyt \rewrite u\subn{x_i}{\sn(\lambda y.\pi_i(t))^i}[\sh (\lambda y.t):1,\dots,n]$
\quad


\begin{proof}
Analogous to the proof of Lemma~\ref{lem:unsharing}.
\end{proof}


\begin{ALlemma}
\label{lem:lift-sn}
For any terms $u$ and $u'$ if $u\rewrite u'$, then $\sn(u)
~\rewrite^\star~ \sn(u')$. Moreover, for any terms of multiplicity
$n$, $t$ and $t'$, and any $i\leq n$, if $t\rewrite t'$, then
$\sn(\pi_i(t))\rewrite^\star\sn(\pi_i(t'))$.
\end{ALlemma}


\begin{proof}
The following three cases are proven simultaneously.
\\ 1) If $u ~\ALnotbeta~ u'$, then $\sn(u) ~=~ \sn(u')$.
\\ 2) If $u ~\ALbeta~ u'$, then the statement follows by structural induction on $u$.
\\ 3) If $t=\tup t\G ~\rewrite~ t'=\tup{t'}\G'$, by case 2 (and Proposition~\ref{prop:sn_pi}) the result follows from the reduction
%
	$t_i\share*{}{x_1}\dots\share*{}{x_m}\G\rewrite
	 t'_i\share*{}{x_1}\dots\share*{}{x_m}\G'$,
which will be shown next.

If the reduction happens inside $\vv t$, or inside $\G$ without substituting into
$\vv t$, the result is immediate.
%
Otherwise, write
\[
	t'=\tup t\sub{y_{1,1}}{u_{1,1}}\sub{y_{m,k_m}}{u_{m,k_m}}\G'
\]
where $\FV(t_i)=\{y_{i,1},...,y_{i,k_i}\}$.
%
Without loss of generality, consider $i=1$. Then
%
\[
\begin{aligned}
\pi_1(t)
	&= t_1\share*{}{y_{1,1}}\dots\share*{}{y_{m,k_m}}\G
\\ &\rewrite t_1\sub{y_{1,1}}{u_{1,1}}\sub{y_{1,k_1}}{u_{1,k_1}}\share*{}{u_{2,1}}\dots\share*{}{u_{m,k_m}}\G'
\\ &\rewrite t_1\sub{y_{1,1}}{u_{1,1}}\sub{y_{1,k_1}}{u_{1,k_1}}\share*{}{z_1}\dots\share*{}{z_l}\G'
\\ &= \pi_1(t')\;,
\end{aligned}
\]
%
where $\{z_1, \dots, z_l\}$ are the free variables of $u_{2,1}$,
$\dots$, $u_{m,k_m}$.

\end{proof}


\medskip
\noindent
\textbf{Lemma~\ref{lem:Red-AddSharings}.}
\quad
For any formula $B$,  if $(u)\overline{w}\in |B|$ then $(u\share*{\vv*x}y)\overline{w} \in |B|$.

\begin{proof}
 We proceed by induction on $B$.
\\
1) $B$ is a variable. Then $|B| = \SN$ and the result is given by Lemma~\ref{lem:SN-AddSharings}.
\\
2) $B= C\vlim D$.
Suppose $(u)\overline{w} \in |C\vlim D|$. Let $t\in|C|$. We prove that  $((u\share xx)\overline{w})t \in |D|$. Because $(u)\overline{w} \in |C\vlim D|$ and $t\in|C|$, we have $((u)\overline{w})t \in |D|$ and by the induction hypothesis, $((u\share xx)\overline{w})t \in |D|$. It follows that $(u\share xx)\overline{w} \in |C\vlim D|$.
\\
3) $B= C\cap B$.
The result directly follows from the induction hypothesis using the fact that $|C\cap D| = |C|\cap|D|$.

\end{proof}



\medskip
\noindent
\textbf{Lemma~\ref{lem:IntCaseLambdaSharing}.}
\quad
\begin{compactenum}[\upshape(i)]
%
\item If $(u\{v/x\})\overline{w} \in |B|$ and $v\in\SN$, then $((\lambda x.u) v)\overline{w} \in |B|$.
%
\item If $(u\subn{x_i}{\sn(t)^i}[\sh (t)])\overline{w} \in |B|$ and $t\in\SN$, 
	\\ then $(u[x_1,\dots,x_n \leftarrow t])\overline{w} \in |B|$.
%
\item If $(u\subn{x_i}{\sn(\lambda y.\pi_i(t))^i}[\sh (\lambda y.t)])\overline{w} \in |B|$ and $t\in\SN$,
	\\ then $(u[x_1,\dots,x_n \twoheadleftarrow \lambda y.t])\overline{w} \in |B|$.
%
\end{compactenum}


\begin{proof}
 We proceed by induction on $B$.
\\
1) $B$ is a variable. Then $|B| = \SN$ and the result is given by Lemma~\ref{lem:IntCaseLambda0}.
\\
2) $B= C\vlim D$.
Suppose $(u\{v/x\})\overline{w} \in |C\vlim D|$ and $v\in\SN$. Let $t\in|C|$. We prove that  $(((\lambda x.u) v)\overline{w})t \in |D|$. Because $(u\{v/x\})\overline{w} \in |C\vlim D|$ and $t\in|C|$, we have $((u\{v/x\})\overline{w})t \in |D|$ and by the induction hypothesis, $(((\lambda x.u) v)\overline{w})t \in |D|$. It follows that $((\lambda x.u) v)\overline{w} \in |C\vlim D|$.
%\\
%3) $B= C\cap B$.
%The result directly follows from the induction hypothesis using the fact that $|C\cap D| = |C|\cap|D|$.
\end{proof}
%
% \Comment{For proving Proposition~\ref{prop:IntSubst} (case 3), we need the statement of Lemma~\ref{lem:IntCaseLambda} only in the case $\overline{w}$ is empty. But for the inductive proof of this statement, the general statement of Lemma~\ref{lem:IntCaseLambda} is needed (case 2 of the proof). \\
%   Then for proving case 1 ($|B| = \SN$), we need the more complicated statement of Lemma~\ref{lem:IntCaseLambda0}: the proof by induction of Lemma~\ref{lem:IntCaseLambda0} introduces the case of sharing (case xx of the proof of Lemma~\ref{lem:IntCaseLambda0})  }




\medskip
\noindent
\textbf{Lemma~\ref{prop:IntSubst}.}
\quad
If $(x_i:A_i)_{i\leq n}\vdash_S u:B$ and $v_i \in |A_i|$, then $u\subn{x_i}{v_i} \in |B|$.

\begin{proof}
We prove by induction on $A$ that (i) $ |A| \subseteq  \SN$ and (ii) for each $x\in \VAR$ and $\overline{t} \in \SN^{<\omega}$, $(x)\overline{t} \in |A|$.
\\
1) If $A=X$, then $|A| =\SN$ and the result follows from Lemma~\ref{lem:HeadVar}.
\\
2) Suppose $A=C\vlim D$.
\\
(i) Let $u \in |A|$. For $x \in \VAR$, we have $x\in |C|$ by the induction hypothesis; therefore $(u)x \in |D|$ and by the induction hypothesis $(u)x \in \SN$. It follows  $u \in \SN$.
\\
(ii) Let $x\in \VAR$ and $\overline{t} \in \SN^{<\omega}$. Let $v\in |C|$; then by the induction hypothesis,  $v\in \SN$ and therefore, also by the induction hypothesis, $((x)\overline{t})v \in |D|$. It follows that $(x)\overline{t} \in |C \vlim D|$.
\\
3) Suppose $A=C \cap D$. By definition of $|-|$, we have $u\in |A|$ iff $u\in |C|$ and $u\in |D|$, and the result follows directly by the induction hypothesis.
\end{proof}


\medskip
\noindent
\textbf{Lemma~\ref{lem:HeadVar}.}
\quad
If $x\in \VAR$ and $\overline{t} \in \SN^{<\omega}$, then
$(x)\overline{t} \in \SN$.



\begin{proof}
Every reduction step in $(x)\overline{t}$ can be lifted to a reduction
step in one of the terms of $\overline{t}$, with the exception of
reductions on the form $(u\g)v\ALnotbeta(u)v\g$. The crucial fact is
that if $(u\g)v\ALnotbeta(u)v\g$, then a reduction step in $(u)v\g$
can be lifted to either $u\g$ or $v$, as $\g$ can not capture any
variables in $v$, and hence can not interact with it.
%
As there may only be finitely many reduction steps of a form that
cannot be lifted, the result follows by contradiction.
\end{proof}



\begin{ALlemma}\label{lem:SN-AddSharings}
If $(u)\overline{w} \in \SN$, then $(u\share xx)\overline{w} \in \SN$.
\end{ALlemma}


\begin{proof}
First show the result for empty $\overline{w}$, by contradiction.
%
Let $w_0\rewrite^1 w_1\rewrite^1 \ldots$ be an infinite reduction from $w_0=u\share xx$.
%
Each $w_i$ can be written as $w'_i\G[i]$, where $\G[i]$ the largest possible sequence of sharings of the form $\share xx$ (with the body a variable).
%
A rewrite $w_i\rewrite^1 w_{i+1}$ is of one of the following three forms: \textit{a)} it is inside $w'_i$, \textit{b)} it is inside $\G[i]$, or \textit{c)} it rewrites $w'_i$ to $w'_{i+1}\G'$, so that $\G[i+1]=\G'\G[i]$.
\[
		a)\quad	w'_i\G[i] \rewrite^1 w'_{i+1}\G[i]
\qquad	b)\quad	w'_i\G[i] \rewrite^1 w'_i\G[i+1]
\qquad	c)\quad	w'_i\G[i] \rewrite^1 w'_{i+1}\G'\G[i]
\]
The reduction path from $w_0$ is modified as follows: steps of type \textit{a)} and \textit{c)} are retained (up to renaming of variables), but those of type \textit{b)} are omitted.
%
It will be shown that the resulting path \textit{1)} is well-defined, \textit{2)} is infinite, and \textit{3)} generates an infinite reduction from $u$.


First, note that a rewrite step of type \textit{b)} must be an instance of rule %\eqref{eqn:execute unary substitution} or 
\eqref{eqn:compound sharings}.
%
%In the latter case, 
In this case, $w'_i$ remains unaffected, i.e.\ $w'_i=w'_{i+1}$. %; in the former case, a sharing $\share xy$ is evaluated by a substitution $\sub xy$; then $w'_i$ and $w'_{i+1}$ are identical up to renaming of $x$ by $y$.
%
This proves \textit{1)}.
%
For \textit{2)}, each step of type \textit{b)} reduces the length of $\G[i]$ by 1; thus, if the original infinite reduction from $w_0$ contains infinitely many steps of type \textit{b)}, it must also contain infinitely many of type \textit{c)}.
%
For \textit{3)} it suffices to observe that in the new path no rewrite step affects the sharing $\share xx$, since the latter is part of $\G[0]$; thus, the new reduction path from $w_0$ immediately gives and infinite reduction from $u$.

Lastly, the full result follows as $x$ must be free in $(u\share
xx)\overline{w}$, and hence all but a finite number of reduction steps
may be lifted from $(u\share xx)\overline{w}$ to
$(u)\overline{w}\share xx$.

%\TODO{ write a condensed proof}
%
\end{proof}






\medskip
\noindent
\textbf{Lemma~\ref{lem:IntCaseLambda0}.}
\quad
If $((u\sub xv)\G\G')\overline{w\G} \in \SN$, then $((((\lambda x.u)\G) v)\G')\overline{w\G} \in \SN$.

\begin{proof}
 Let $T = ((((\lambda x.u)\G) v)\G')\overline{w\G}$ and $T' = ((u\sub xv)\G\G')\overline{w\G}$.
\\
In order to show that $T\in\SN$, we show that each one-step reduction from $T$ gives a term $U\in\SN$.
%
We proceed by induction on $Mes(T) = (S(T'), lh(u), lh(v), \lh\g)$ where $\lh\G$ is the length of the sequence $\G$.
%
We consider the different possibilities for $U$.
\smallskip
\\
1)  $U = ((u\{v/x\})\G\G')\overline{w\G}$ (in this case $\G$ is empty).
\\
We have $U=T'\in\SN$.
\smallskip
\\
2)  $U = ((((\lambda x.u^*)\G) v)\G')\overline{w\G}$ with $u\rightsquigarrow^1 u^*$.
\\
Consider $U' = ((u^*\{v/x\})\G\G')\overline{w\G}$. We have $T'\rightsquigarrow^1 U'$ and therefore $U'\in\SN$ and $S(U')<S(T')$. Because $Mes(U) < Mes(T)$, we have by the induction hypothesis, $U\in\SN$.
\smallskip
\\
3)  $U = ((((\lambda x.u_1)\g\G) v)\G')\overline{w\G}$ with $u = u_1\g$ and $x \notin FV(\gamma)$.
\\
Consider $U' = ((u_1\{v/x\})\g\G\G')\overline{w\G}$.
We have
\\
$T' = (((u_1\g)\{v/x\})\G\G')\overline{w\G} = (((u_1\{v/x\})\g)\G\G')\overline{w\G} = U'$ and $U'\in\SN$.
Because $S(U') = S(T')$ and $lh(u_1) < lh(u)$, we have $Mes(U) < Mes(T)$ and by the induction hypothesis, $U\in\SN$.
\smallskip
\\
4)  $U = ((((\lambda x.u)[\Gamma^*]) v)\G')\overline{w\G}$ with $[\Gamma^*]$ is obtained by a one-step reduction in $\G$ using any rule except 
%\eqref{eqn:execute unary substitution}, 
\eqref{eqn:share application} or \eqref{eqn:distributor elimination}.
\\
Consider $U' = ((u\{v/x\})[\Gamma^*]\G')\overline{w\G}$.  We have $T'\rightsquigarrow^1 U'$ and therefore $U'\in\SN$ and $S(U')<S(T')$. Because $Mes(U) < Mes(T)$, we have by the induction hypothesis, $U\in\SN$.
\\
The same argument holds if $U$ is obtained from $T$ by any one-step reduction in $\G'\overline{w\G}$ using any rule except %\eqref{eqn:execute unary substitution}, 
\eqref{eqn:share application} or \eqref{eqn:distributor elimination}.
\smallskip
\\
5)  $U = ((((\lambda x.u^*)[\Gamma^*]) v)\G')\overline{w\G}$ and $U$ is obtained from $T$ by a one-step reduction in $\G$ using one of the rules
% \eqref{eqn:execute unary substitution}, 
\eqref{eqn:share application} and \eqref{eqn:distributor elimination} (these rules may apply substitutions to $u$ transforming it into $u^*$).
\\
Consider $U' = ((u^*\{v/x\})[\Gamma^*]\G')\overline{w\G}$. $U'$ is obtained from $T'$ by applying the same rule $\G$ and therefore $U'\in\SN$ and $S(U')<S(T')$. Because $Mes(U) < Mes(T)$, we have by the induction hypothesis, $U\in\SN$.
\\
The same argument holds if $U$ is obtained from $T$ by any one-step reduction in $\G'\overline{w\G}$ using one of the rules
 %\eqref{eqn:execute unary substitution}, 
\eqref{eqn:share application} and \eqref{eqn:distributor elimination} (in this case the substitution may transform also $u$, $v$ and $\G$).
\smallskip
\\
6)  $U = ((((\lambda x.u)[\Sigma]) v)\g\G')\overline{w\G}$ with $\G = [\Sigma]\g$ (application of rule \eqref{eqn:sharing above application function})
\\
Consider $U' = ((u\{v/x\})[\Sigma]\g\G')\overline{w\G}$.
We have $T' = U'$ and therefore $U'\in\SN$ and $S(U') = S(T')$. Because $lh([\Sigma]) < \lh\G$, we have $Mes(U) < Mes(T)$ and by the induction hypothesis, $U\in\SN$.
\smallskip
\\
7)  $U = ((((\lambda x.u)\G) v_1)\g\G')\overline{w\G}$ with $v = v_1\g$.
\\
Consider $U' = ((u\{v_1/x\})\G\g\G')\overline{w\G}$.
We have $T' = (((u)\{v_1\g/x\})\G\G')\overline{w\G}$. $T'$ reduce to $U'$ in $n\ge 0$ steps (it can be $0$ in the case where $u = x$). Therefore $U'\in\SN$ and $S(U') \leq S(T')$. Because $lh(v_1) < lh(v)$, we have $Mes(U) < Mes(T)$ and by the induction hypothesis, $U\in\SN$.
\smallskip
\\
8)  $U = ((((\lambda x.u)\G) v*)\G')\overline{w\G}$ with $v\rightsquigarrow^1 v^*$.
\\
Consider $U' = ((u\{v^*/x\})\G\G')\overline{w\G}$. We have $T'\rightsquigarrow^1 U'$ (because $x$ has exactly one occurrence in $u$) and therefore $U'\in\SN$ and $S(U')<S(T')$. Because $Mes(U) < Mes(T)$, we have by the induction hypothesis, $U\in\SN$.

\end{proof}


%
%Proof of the remaining cases of Lemma~\ref{lem:safe reflection}
%
%\begin{proof}
%\begin{itemize}
%
%	\item[\eqref{eqn:sharing above abstraction}]
%There are two cases:
%\begin{itemize}
%
%	\item
%Given $\Gamma,\Delta \vdash_D (\lambda x.t)\share*{\vv*y}v:A$, we have $\Gamma,x:B,\vv*y:C\vdash_D t:A$ and $\Delta \vdash_D v:C$, which gives $\Gamma,\Delta\vdash_D \lambda x.(t\share*{\vv*y}v):A$.
%
%	\item
%Given $\Gamma,\Delta \vdash_D (\lambda x.t)\distr*{\vv*y}zv:A$, we have $\Gamma,x:B,\vv*y:\bigcap_{i\leq n}(C_i\vlim D_i)\vdash_D t:A$ and for each $i\leq n$, $\Delta,z:C^\star_i \vdash_D v:D_i$, which gives $\Gamma,\Delta\vdash_D \lambda x.(t\distr*{\vv*y}zv):A$.
%\end{itemize}
%
%%\label{eqn:sharing above application function}
%%	(u\g)t & ~\ALnotbeta~ ((u)t)\g
%
%	\item[\eqref{eqn:sharing above application function}]
%There are two cases:
%\begin{itemize}
%
%	\item
%Given $\Gamma,\Delta,\Sigma \vdash_D ((u)t)\share*{\vv*y}v:A$. As this is a redex of the given rule $\vv*y$ binds only in $u$ and
%we have $\Gamma,\vv*y:C\vdash_D u:B\vlim A$, $\Delta\vdash_D t:B$ and
%$\Sigma \vdash_D v:C$, which gives $\Gamma,\Delta,\Sigma\vdash_D (u\share*{\vv*y}v)t:A$.
%
%	\item
%Given $\Gamma,\Delta,\Sigma \vdash_D ((u)t)\distr*{\vv*y}zv:A$. As this is a redex of the given rule $\vv*y$ binds only in $u$ and
%we have $\Gamma,\vv*y:\bigcap_{i\leq n}(C_i\vlim D_i)\vdash_D u:B\vlim A$, $\Delta\vdash_D t:B$ and,
%for each $i\leq n$, $\Sigma,z:C^\star_i \vdash_D v:D_i$, which gives $\Gamma,\Delta,\Sigma\vdash_D (u\distr*{\vv*y}zv)t:A$.
%\end{itemize}
%
%%\label{eqn:sharing above application argument}
%%	(u)t\g & ~\ALnotbeta~ ((u)t)\g
%
%	\item[\eqref{eqn:sharing above application argument}]
%There are two cases:
%\begin{itemize}
%
%	\item
%Given $\Gamma,\Delta,\Sigma \vdash_D ((u)t)\share*{\vv*y}v:A$. As this is a redex of the given rule $\vv*y$ binds only in $t$ and
%we have $\Gamma\vdash_D u:B\vlim A$, $\Delta,\vv*y:C\vdash_D t:B$ and
%$\Sigma \vdash_D v:C$, which gives $\Gamma,\Delta,\Sigma\vdash_D (u)t\share*{\vv*y}v:A$.
%
%	\item
%Given $\Gamma,\Delta,\Sigma \vdash_D ((u)t)\distr*{\vv*y}zv:A$. As this is a redex of the given rule $\vv*y$ binds only in $t$ and
%we have $\Gamma \vdash_D u:B\vlim A$, $\Delta,\vv*y:\bigcap_{i\leq n}(C_i\vlim D_i)\vdash_D t:B$ and,
%for each $i\leq n$, $\Sigma,z:C^\star_i \vdash_D v:D_i$, which gives $\Gamma,\Delta,\Sigma\vdash_D (u)t\distr*{\vv*y}zv):A$.
%\end{itemize}
%
%%\label{eqn:sharing above sharing}
%%	u\share x{t\g} & ~\ALnotbeta~ u\share xt\g
%
%	\item[\eqref{eqn:sharing above sharing}]
%There are two cases:
%\begin{itemize}
%
%	\item
%Given $\Gamma,\Delta,\Sigma^+ \vdash_D u\share*{\vv*x}t\share*{\vv*y}v:A$. As this is a redex of the given rule $\vv*y$ binds only in $t$. We have $\Gamma,\vv*x:B\vdash_D u:A$, $\Delta,\vv*y:C\vdash_D t:B$ and $\Sigma^+ \vdash_D v:C$. In case $u$ is an $n$-term and $\vv*y$ is the empty sequence, we may have $\Sigma^+\neq\Sigma$, in which case we also have $\Sigma\vdash_D v:C$, which gives $\Gamma,\Delta,\Sigma\vdash_D u\share*{\vv*x}{t\share*{\vv*y}v}:A$.
%
%	\item
%Given $\Gamma,\Delta,\Sigma \vdash_D u\share*{\vv*x}t\distr*{\vv*y}zv:A$. As this is a redex of the given rule $\vv*y$ binds only in $t$ and we have $\Gamma,\vv*x:B \vdash_D u:A$, $\Delta,\vv*y:\bigcap_{i\leq n}(C_i\vlim D_i)\vdash_D t:B$ and,
%for each $i\leq n$, $\Sigma,z:C^\star_i \vdash_D v:D_i$, which gives $\Gamma,\Delta,\Sigma\vdash_D u\share*{\vv*x}{t\distr*{\vv*y}zv}:A$.
%\end{itemize}
%
%%\label{eqn:sharing above distribution}
%%	u\distr xy{t\g} & ~\ALnotbeta~ u\distr xyt\g
%%		& \text{if}~y\in\FV(t) &&&
%
%	\item[\eqref{eqn:sharing above distribution}]
%There are two cases:
%\begin{itemize}
%
%	\item
%Given $\Gamma,\Delta,\Sigma^+ \vdash_D u\distr*{\vv*x}qt\share*{\vv*y}v:A$. As this is a redex of the given rule $\vv*y$ binds only in $t$ and we have $\Gamma,\vv*x:\bigcap_{i\leq n}(E_i\vlim B_i)\vdash_D u:A$, for each $i\leq n$, $\Delta,\vv*y:C,q:E^\star_i\vdash_D t:B_i$ and
%$\Sigma^+ \vdash_D v:C$. In case $u$ is an $n$-term and $\vv*y$ is the empty sequence, we may have $\Sigma^+\neq\Sigma$,
%in which case we also have $\Sigma\vdash_D v:C$,
%which gives $\Gamma,\Delta,\Sigma\vdash_D u\distr*{\vv*x}q{t\share*{\vv*y}v}:A$.
%
%	\item
%Given $\Gamma,\Delta,\Sigma \vdash_D u\distr*{\vv*x}qt\distr*{\vv*y}zv:A$. As this is a redex of the given rule $\vv*y$ binds only in $t$ and we have $\Gamma,\vv*x:\bigcap_{i\leq n}(E_i\vlim B_i) \vdash_D u:A$, for each $i\leq n$, $\Delta,\vv*y:\bigcap_{j\leq m}(C_j\vlim D_j),q:E^\star_i\vdash_D t:B$ and, for each $j\leq m$, $\Sigma,z:C^\star_j \vdash_D v:D_j$, which gives $\Gamma,\Delta,\Sigma\vdash_D u\distr*{\vv*x}q{t\distr*{\vv*y}zv}:A$.
%\end{itemize}
%
%%\label{eqn:execute unary substitution}
%%	u\share*xt~\ALnotbeta~u\sub xt
%
%%	\item[\eqref{eqn:execute unary substitution}]
%%Given $\Gamma,\Delta \vdash_D u\sub xt:A$, we have by Proposition~\ref{prop:typing substitution}, some $B$ such that $\Gamma, x:B \vdash_D u:A$ and $\Delta\vdash_D t:B$, which gives $\Gamma,\Delta\vdash_D u\share*xt$.
%
%%\label{eqn:compound sharings}
%%	u\share*{\vv*x}{y_i}\share yt ~\ALnotbeta~
%%	u\share*{y_1,\dots,y_{i-1},\vv*x,y_{i+1},\dots,y_n}{t}
%
%	\item[\eqref{eqn:compound sharings}]
%Given $\Gamma,\Delta\vdash_D u\share*{y_1,\dots,y_{i-1},\vv*x,y_{i+1},\dots,y_n}{t}:A$, we have $\Gamma,y_1,\dots,y_{i-1},\vv*x,y_{i+1},\dots,y_n:B\vdash_D u:A$ and $\Delta\vdash_D t:B$. This gives $\Gamma,\vv*y:B\vdash_D u\share*{\vv*x}{y_i}$, which again gives $\Gamma,\Delta\vdash_D u\share*{\vv*x}{y_i}\share*{\vv*y}t$.
%
%%\label{eqn:share application}
%%	u\share x{(v)t} ~\ALnotbeta~
%%	u\subn{x_i}{(y_i)z_i}\share yv\share zt
%
%	\item[\eqref{eqn:share application}] 
%Where $n\geq1$. Given $\Gamma,\Delta,\Sigma \vdash_D u\subn{x_i}{(y_i)z_i}\share yv\share zt:A$. As this is a redex of the given rule, $\vv*z$ binds only in $u$,
%we have $\Gamma,\vv*y:D,\vv*z:C\vdash_D u\subn{x_i}{(y_i)z_i}:A$;
%and by Proposition~\ref{prop:typing substitution}, there is an $B$, such that $D=C\vlim B$.
%We then have $\Delta\vdash_D v:C\vlim B$ and $\Sigma\vdash_D t:C$, which gives $\Gamma,\Delta,\Sigma\vdash_D u\share*{\vv*x}{(v)t}:A$.
%
%\end{itemize}
%\end{proof}
%

%
%
%\begin{ALlemma}
%Subject reduction for the distributor rewrite rules of $D_a$.
%\end{ALlemma}
%
%\begin{proof}
%%
%Rule~\eqref{eqn:share abstraction} (distributor introduction), before:
%\[
%\vlderivation{
%  \vliiin{}{}
%	{\vdash u\share*{\vv*x}{\lambda y.t}:C}
%	{\vlhy{\vv*x:\bigcap_{j\leq m}(A'_j\vlim B'_j)\vdash u:C}}
%	{\vlhy{}}
%	{\vlin{}{}
%	  {\vdash\lambda y.t:\bigcap_{j\leq m}(A_j\vlim B_j)}
%	  {\vlhy{
%	    \left(
%		\vlinf{}{}
%	 	  {\vdash\lambda y.t:A_j\vlim B_j}
%		  {y:A_j\vdash t:B_j}
%		\right)_{j\leq m}
%	}}}
%}
%\]
%
%\noindent
%Rule~\eqref{eqn:share abstraction} (distributor introduction), after:
%\[
%  \vliiinf{}{}
%	{u\distr*{\vv*x}y{\tup z\share zt}:C}
%	{\raisebox{-25pt}{
%		$\vv*x:\bigcap_{j\leq m}(A'_j\vlim B'_j)\vdash u:C$
%	}}	
%	{\vlhy{}}
%	{\left(\vlderivation{
%		\vliiin{}{}
%	 	  {y:A_j\vdash \tup z\share zt : B'_j\vlan\dotso\vlan B'_j}
%		  {\vlin{}{}
%			{(z_i:B'_j)_{i\leq n}\vdash \tup z:B'_j\vlan\dotso\vlan B'_j}
%			{\vlhy{
%				\Big(
%					\raisebox{5pt}{$\vlinf{}{}{z_i:B'_j\vdash z_i:B'_j}{}$}
%				\Big)_{i\leq n}}}
%		  }
%		  {\vlhy{}}
%		  {\vlhy{y:A_j\vdash t:B_j}}
%		}
%		\right)_{j\leq m}
%	}
%\]
%
%\noindent
%Rule~\eqref{eqn:distributor elimination} (distributor elimination), before:
%\[
%\vliiinf{}{}
%  {u\distr xy{\tup t\share*{\vv*z}y}:C}
%  {\raisebox{-18pt}{$(x_i:\bigcap_{j\leq m}(A'_j\vlim B^i_j))_{i\leq n}\vdash u:C$}}
%  {}
%  {\left(
%	\vlderivation{
%	  \vliiin{}{}
%		{y:A_j\vdash \tup t\share*{\vv*z}y:B^1_j\vlan\dotso\vlan B^n_j}
%		{\vlin{}{}
%		  {\vv*z:A_j\vdash\tup t:B^1_j\vlan\dotso\vlan B^n_j}
%		  {\vlhy{(\vv*z_i:A_j\vdash t_i:B^i_j)_{i\leq n}}}
%		}
%		{\vlhy{}}
%		{\vlin{}{}{y:A_j\vdash y:A_j}{}}
%	}
%   \right)_{j\leq m}
%  }
%\]
%
%\noindent
%Rule~\eqref{eqn:distributor elimination} (distributor elimination), after: for each $i\leq n$:
%\[
%\vlinf{}{}
%  {\vdash\lambda y_i.t_i\share*{\vv*z_i}{y_i}:\bigcap_{j\leq m}(A'_j\vlim B^i_j)}
%  {\left(
%	\vlderivation{
%	  \vlin{}{}
%		{\vdash\lambda y_i.t_i\share*{\vv*z_i}{y_i}:A'_j\vlim B^i_j}
%		{\vliiin{}{}
%		  {y_i:A'_j\vdash t_i\share*{\vv*z_i}{y_i}:B^i_j}
%		  {\vlhy{\vv*z_i:A_j\vdash t_i:B^i_j}}
%		  {\vlhy{}}
%		  {\vlin{}{}{y_i:A'_j\vdash y_i:A'_j}{}}
%	}}
%   \right)_{j\leq m}
%  }
%\]
%%
%Note that the type assignments $\vv*z_i:A_j$ and $y_i:A'_j$ may be combined in the above due to the fact that $A_j=A'_j$ or $\vv*z_i$ is empty.
%
%
%
%\noindent
%During reduction,
%% in the typing derivations for $t^n$ in
%%\[
%%\vliiinf{}{}
%%	{u\distr xy{t^n}:C}
%%	{(x_i:\bigcap_{j\leq m}(A_j\vlim B^i_j))_{i\leq n}\vdash u:C}	
%%	{\vlhy{}}
%%	{(y:V_j\vdash t^n:B^1_j\vlan\dotso\vlan B^n_j)_{j\leq m}}
%%\]
%the following holds for the leading variables in the terms concerned:
%\begin{itemize}
%
%	\item
%if a variable is weakened, it remains weakened;
%
%	\item
%if it is not, it may become weakened only through a beta-step in $t^n$;
% 
%	\item
%if it becomes weakened, it may retain its previous type.
%
%\end{itemize}
%
%\end{proof}
%
%
%
%\newpage
%
%\noindent
%\begin{ALlemma}
%Reverse subject reduction for the distributor rewrite rules of $D_a$.
%\end{ALlemma}
%
%\begin{proof}
%\noindent
%Rule~\eqref{eqn:share abstraction} (distributor introduction), after:
%\[
%  \vliiinf{}{}
%	{u\distr*{\vv*x}y{\tup z\share zt}:C}
%	{\raisebox{-25pt}{
%		$(x_i:\bigcap_{j\leq m}(A'_j\vlim B'_j))_{i\leq n}\vdash u:C$
%	}}	
%	{\vlhy{}}
%	{\left(\vlderivation{
%		\vliiin{}{}
%	 	  {y:A_j\vdash \tup z\share zt : B'_j\vlan\dotso\vlan B'_j}
%		  {\vlin{}{}
%			{(z_i:B'_j)_{i\leq n}\vdash \tup z:B'_j\vlan\dotso\vlan B'_j}
%			{\vlhy{
%				\Big(
%					\raisebox{5pt}{$\vlinf{}{}{z_i:B'_j\vdash z_i:B'_j}{}$}
%				\Big)_{i\leq n}}}
%		  }
%		  {\vlhy{}}
%		  {\vlhy{y:A_j\vdash t:B_j}}
%		}
%		\right)_{j\leq m}
%	}
%\]
%
%% NOTE: Here we need the full definition of replacement types,
%%       since the B'_j in the types for the x_i are intersections;
%%       otherwise we must show we can adjust the type of the x_i
%%
%%Where $B_j=\bigcap_{i\leq p}B^i_j$ and $B'_j=\bigcap_{i\leq p}B'^i_j$ with each $B'^i_j$ a replacement type for $t:B^i_j$.
%%%
%%Then the statement $t:B_j$ must have been derived by:
%%%
%%\[
%%  \vlinf{}{}
%%    {y:A_j\vdash t:B_j}
%%    {(y:A_j\vdash t:B^i_j)_{i\leq p}}
%%\]
%
%
%\noindent
%Rule~\eqref{eqn:share abstraction} (distributor introduction), before, is as follows:
%\[
%\vlderivation{
%  \vliiin{}{}
%	{\vdash u\share*{\vv*x}{\lambda y.t}:C}
%	{\vlhy{\vv*x:\bigcap_{j\leq m}(A'_j\vlim B'_j)\vdash u:C}}
%	{\vlhy{}}
%	{\vlin{}{}
%	  {\vdash\lambda y.t:\bigcap_{j\leq m}(A_j\vlim B_j)}
%	  {\vlhy{
%	    \left(
%		\vlinf{}{}
%	 	  {\vdash\lambda y.t:A_j\vlim B_j}
%		  {y:A_j\vdash t:B_j}
%		\right)_{j\leq m}
%	}}}
%}
%\]
%
%%\[
%%\vlderivation{
%%  \vliiin{}{}
%%	{\vdash u\share*{\vv*x}{\lambda y.t}:C}
%%	{\vlhy{\vv*x:\bigcap^{i\leq n}_{j\leq m}(A'_j\vlim B'^i_j)\vdash u:C}}
%%	{\vlhy{}}
%%	{\vlin{}{}
%%	  {\vdash\lambda y.t:\bigcap^{i\leq n}_{j\leq m}(A_j\vlim B_j)}
%%	  {\vlhy{
%%	    \left(
%%		\vlinf{}{}
%%	 	  {\vdash\lambda y.t:A_j\vlim B_j}
%%		  {y:A_j\vdash t:B_j}
%%		\right)_{\overset{\scriptstyle {i\leq n}}{j\leq m}}
%%	}}}
%%}
%%\]
%
%
%\bigskip
%
%\noindent
%Rule~\eqref{eqn:distributor elimination} (distributor elimination), after: consider the following term: $u\subn{x_i}{\lambda y_i.t_i\share*{\vv*z_i}{y_i}}$.
%%
%For $u$, by Proposition~\ref{prop:typing substitution} we get the typing statement below left, from which the one below right is derived using  and Proposition~\ref{prop:intersection weakening}. 
%
%\[
%\textstyle
%	(x_i:B_i\vlim C_i)_{i\leq n}\vdash_D u:A
%\quads3
%	\vv*x:\bigcap_{j\leq n}(B_j\vlim C_j)\vdash_D u:A
%\]
%For $t_i$ we get:
%\[
%\vlderivation{
%  \vlin{}{}
%	{\vdash\lambda y_i.t_i\share*{\vv*z_i}{y_i}:B_i\vlim C_i}
%	{\vliiin{}{}
%	  {y:B_i\vdash t_i\share*{\vv*z_i}{y_i}\:C_i}
%	  {\vlhy{\vv*z_i:B_i\vdash t_i:C_i}}
%	  {\vlhy{}}
%	  {\vlin{}{}{y:B_i\vdash y:B_i}{}}
%}}
%\]
%%
%Since all $t_j$ are variants, to obtain the desired typing for the tuple $\tup t$, we first replicate the typing derivations for a given $t_j$ (below left), after which a renaming of variables gives the desired typing derivation (below right).
%\[
%  \vlderivation{
%	\vlin{}{}
%		  {\vv*z^{\kern2pt i}:B_j\vdash\tup^{t_j}:C_j\vlan\dotso\vlan C_j}
%		  {\vlhy{(\vv*z_j^{\kern2pt i}:B_j\vdash t_j^{\kern2pt i}:C_j)_{i\leq n}}}
%		}
%\qquad
%  \vlderivation{
%	\vlin{}{}
%	  {\vv*z:B_j\vdash\tup t:C_j\vlan\dotso\vlan C_j}
%	  {\vlhy{(\vv*z_i:B_j\vdash t_i:C_j)_{i\leq n}}}
%	}
%\]
%Rule~\eqref{eqn:distributor elimination} (distributor elimination), before:
%\[
%\vliiinf{}{}
%  {u\distr*{\vv*x}y{\tup t\share*{\vv*z}y}:A}
%  {\raisebox{-10pt}{$\vv*x:\bigcap_{j\leq n}(B_j\vlim C_j)\vdash_D u:A$}}
%  {}
%  {\left(
%	\vlderivation{
%	  \vliiin{}{}
%		{y:B_j\vdash \tup t\share*{\vv*z}y:C_j\vlan\dotso\vlan C_j}
%		{\vlhy{\vv*z:B_j\vdash\tup t:C_j\vlan\dotso\vlan C_j}}
%		{\vlhy{}}
%		{\vlin{}{}{y:B_j\vdash y:B_j}{}}
%	}
%   \right)_{j\leq m}
% }
%\]
%
%\end{proof}
%
%
%
%
%
%% %=================================
%%
%% \subsection{Preservation of strong normalisation}
%%
%%
%% Figure~\ref{fig:lambda intersection types} presents a calculus of intersection types for the standard lambda-calculus.
%% %
%% The treatment of contexts $\Gamma$ is chosen to allow a close correspondence with derivations in $D_a$.
%%
%%
%% \begin{ALdefinition}
%% The type system $D$ of intersection types for the standard lambda-calculus $\Lambda$ is given by the rules in Figure~\ref{fig:lambda intersection types}.
%% \end{ALdefinition}
%%
%%
%% The type system characterises precisely the strongly normalisable terms.
%%
%%
%% \begin{ALtheorem}[\cite{??}]\label{thm:lambda intersection SN}
%% A lambda term $N\in\Lambda$ is SN if and only if $N$ is typeable in $D$.
%% \end{ALtheorem}
%%
%% \begin{figure}[!tp]
%% \[
%%   \begin{array}{c@{\quads3}c@{\quads3}c}
%% 	  \vlinf{}{\typevar}{x:A\vdash x:A}{}
%% 	&
%% 	  \vlinf{}{\typeabs}
%% 	   {\Gamma \vdash \lambda x.N : A \vlim B}
%% 	   {\Gamma, x : A,\dotsc, x : A \vdash N : B}
%% 	&
%% 	  \vliiinf{}{\typeapp}
%% 	   {\Gamma,\Delta \vdash (N)M : B}
%% 	   {\Gamma \vdash N : A\vlim B}
%% 	   {}
%% 	   {\Delta \vdash M : A}
%% 	\\ \\ \\
%%       \vlinf{}{\cap_{e1}}
%% 	   {\Gamma \vdash N : A}
%% 	   {\Gamma \vdash N : A\cap B}
%% 	&
%%       \vlinf{}{\cap_{e2}}
%% 	   {\Gamma \vdash N : B}
%% 	   {\Gamma \vdash N : A\cap B}
%% 	&
%% 	  \vliiinf{}{\cap_i}
%% 	   {\Gamma \vdash N : A\cap B}
%% 	   {\Gamma \vdash N : A}
%% 	   {}
%% 	   {\Gamma \vdash N : B}
%%   \end{array}
%% \]
%% \caption{Type system $D$ for the lambda-calculus}
%% \label{fig:lambda intersection types}
%% \end{figure}
%%
%%
%% \begin{ALproposition}\label{prop:intersection types preserved}
%% If $N:A$ in $D$ then $\tercoden N:A$ in $D_a$.
%% \end{ALproposition}
%%
%% \begin{proof}
%% %By induction on the typing derivation for $\Gamma\vdash N:A$.
%% %
%% %The case where consists of a single axiom is immediate, and if the last rule in the derivation is one of $\typeapp$, $\cap_{e1}$, $\cap_{e2}$, or $\cap_i$, the induction hypothesis applies immediately.
%% %
%% A typing derivation for $\Gamma \vdash N : A$ translates directly to one for $\Gamma\vdash \tercoden N : A$, with the following notes:
%% %
%% 1) the derivation for $\tercoden N$ ends in an additional series of $(\typeshare)$-inferences as displayed below left; and
%% %
%% 2) a $(\lambda)$-inference in the derivation for $N$ translates to a consecutive $(\lambda)$-inference and $(\typeshare)$-inference, as illustrated below right.
%% %
%% \[
%% 	{\vliiinf{}{\typeshare}
%% 	   {\Gamma, x_i : B  \vdash u\share*{x^1_i,\dotsc,x^n_i}{x_i} : A}
%% 	   {\Gamma, x^1_i \dots x^n_i : B \vdash u : A }
%% 	   {}
%% 	   {x_i : B \vdash x_i : B }
%% 	}
%% \qquad
%% 	\vlderivation{
%% 	 \vlin{}{\lambda}
%% 	  {\Gamma \vdash \lambda x.u\share*{x^1,\dotsc,x^n}x : A\to B}
%% 	  {\vliiin{}{\typeshare}
%% 	   {\Gamma, x : B  \vdash u\share*{x^1,\dotsc,x^n}x : A}
%% 	   {\vlhy{ \Gamma, x^1 \dots x^n : B \vdash u : A }}
%% 	   {\vlhy{\ }}
%% 	   {\vlhy{ x : B \vdash x : B }}
%% 	  }
%% 	}
%% \quad
%% \]
%% \end{proof}
%%
%%
%% \begin{ALtheorem}[PSN]
%% If $N$ is SN then $\tercoden N$ is SN.
%% \end{ALtheorem}
%%
%% \begin{proof}
%% If $N$ is SN then by Theorem~\ref{thm:lambda intersection SN} it is typeable in $D$; then by Proposition~\ref{prop:intersection types preserved} $\tercoden N$ is typeable in $D_a$, and by Theorem~?? $\tercoden N$ is SN.
%% \end{proof}
%
%
%% Proof of Lemma~\ref{lem:weakening RED}
%%
%%\textcolor{blue}{
%%\begin{proof}
%%The proof is by induction on $B$.
%%%
%%The case where $B$ is an atom is covered by Lemma~\ref{lem:weakening SN}; the case $B=C\cap D$ is immediate.
%%%
%%For $B=C\vlim D$, let $v\in\SN$. 
%%%
%%We have to show that $(t\share*{}v)\overline w\in|C\vlim D|$, i.e.\ that $(t\share*{}v)\overline ww\in |D|$ for all $w\in|C|$.
%%%
%%Let $w\in|C|$.
%%%
%%Because $(t)\overline w\in|C\vlim D|$, we have $(t)\overline ww\in|D|$, and by the induction hypothesis, $(t\share*{}v)\overline ww\in|D|$.
%%\end{proof}
%%}
%
%
%
