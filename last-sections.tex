




% =================================================================================================
\section{Characterization of Strongly Normalising Atomic Lambda Terms via Typing}
\label{sec:SND}

In this section we show that the  strongly normalising atomic $\lambda$-terms can be characterized as those typable in a system $D_a$ with intersection types. As for lambda-calculus $D_a$ extends the corresponding simple type system $S_a$ with the natural rules of introduction and elimination of $\cap$. But in addition, the rules for sharing and distributor are  extended in order to take into a account a specificity of atomic lambda-calculus, namely the fact that contracty to lambda-calculus, garbage is part of the computation. This impose to relax the type constraints of abstractions over this pieces of terms.


\subsection{Typing rules for $D_a$}

\begin{ALdefinition}
A variable $y$ occurs \emph{weakened} in a term $t$ if (i) $t$ has a subterm $tv\share*{}y$ such that $y\in\FV(u)$ or 
(ii) $t$ has a subterm $v\share xu$, with $n>0$, such that $y\in\FV(u)$ and each $x_i$ occurs weakened in $v$.
\end{ALdefinition}


\begin{ALlemma}
\label{lem:unfolding weakened variable}
If a variable $y$ occurs free and weakened in a term $t$ then the unfolding of $t$ is of the form $t'\share*{}y$.
\end{ALlemma}

\noindent
The typing rules for $D_a$ are the following:
\[
	\vliinf{}{}
	  {\Gamma \vdash u: A\cap B}
	  {\Gamma \vdash u: A}
	  {\Gamma \vdash u: B}
	\qquad
	\vlinf{}{}
	  {\Gamma \vdash u: A}
	  {\Gamma \vdash u: A\cap B}
	\qquad
	\vlinf{}{}
	  {\Gamma \vdash u: B}
	  {\Gamma \vdash u: A\cap B}
\]

 \noindent 
\textcolor{red}{[rule for sharing to be added]}

\[
	\vliiinf{}{\typedist_{D_a}}
	  {\Gamma,\Delta \vdash u^*\distr xy{t^n}:A}
	  {\Gamma, (x_i:\bigcap_{j\leq m}(B_j\vlim C^i_j))_{i\leq n} \vdash u^*:A}
	  {}
	  {(\Delta, y:V_j\vdash t^n : \vls(C_j^1.\dots.C_j^n))_{j\leq m}}
\]
where in the last rule, $V_j$ may be any type if $y$ occurs \emph{weakened}, as defined below, and $V_j=B_j$ otherwise.


\subsection{Strong Normalisation for $D_a$}

We extend the notion of reducibility set to intersection types by $|A\cap B|=|A|\cap|B|$ and generalise proposition \ref{prop:IntSubst} to the following proposition from which strong normalisation for $D_a$ follows immediately:

                                                                
\begin{ALproposition}\label{prop:IntSubstIntersection}
If $(x_i:A_i)_{i\leq n}\vdash_{D_a} u:B$ and $v_i \in |A_i|$, then $u\subn{x_i}{v_i} \in |B|$.
\end{ALproposition}


\begin{ALlemma}
\label{lem:weakening SN}
If $(t)\overline w\in\SN$, then $\forall v\in\SN.~(t\share*{}v)\overline w\in\SN$.
\end{ALlemma}

\begin{ALlemma}
\label{lem:weakening RED}
If $(t)\overline w\in|B|$, then $\forall v\in\SN.~(t\share*{}v)\overline w\in|B|$.
\end{ALlemma}

\textcolor{blue}{
\begin{proof}
The proof is by induction on $B$.
%
The case where $B$ is an atom is covered by Lemma~\ref{lem:weakening SN}; the case $B=C\cap D$ is immediate.
%
For $B=C\vlim D$, let $v\in\SN$. 
%
We have to show that $(t\share*{}v)\overline w\in|C\vlim D|$, i.e.\ that $(t\share*{}v)\overline ww\in |D|$ for all $w\in|C|$.
%
Let $w\in|C|$.
%
Because $(t)\overline w\in|C\vlim D|$, we have $(t)\overline ww\in|D|$, and by the induction hypothesis, $(t\share*{}v)\overline ww\in|D|$.
\end{proof}
}


\begin{proof}[of proposition \ref{prop:IntSubstIntersection}]
%
We proceed by induction on the derivation of $(x_i:A_i)_{i\leq n}\vdash_D u:B$. The rules of $S_a$ are treated as in the proof of Proposition~\ref{prop:IntSubst}. The cases of the 
intersection rules are immediate. The rules for sharing and distributor need a specifice argument in the case of an abstration on a weakened variable. Suppose that last rule is
\[
\hspace{-\leftmargin}
  \vliiinf{}{\typedist_{D_a}}
	{(y_i:C_i)_{i\leq k}, (z_i:D_i)_{i\leq m} \vdash s\distr xy{u^n} : C}
	{(y_i:C_i)_{i\leq k}, (x_i:\bigcap_{j\leq m}(A_j \vlim B_j))_{i\leq n} \vdash s : C}
	{\hspace{-6pt}}
	{((z_i:D_i)_{i\leq m}, y : A_j \vdash u^n : B_j\vlan\cdots\vlan B_j)_{j\leq m}}
\]
and that $y$ occurs weakened in $u^n$. Let $v_i\in|C_i|$ and $w_j\in|D_j|$ for $i\leq k$ and $j\leq m$, and let $s'=s\subn[k]{y_i}{v_i}$ and $u'=u^n\subn[m][j]{z_j}{w_j}$.
%
We have to prove that the following term is in $|C|$:
\[
	(s\distr xy{u^n})\subn[k]{y_i}{v_i}\subn[m][j]{z_j}{w_j}=s'\distr xy{u'}\;.
\]
%
By the induction hypothesis, for each $j\leq m$, $u'=u^n\subn[m][j]{z_j}{w_j}\in|B_j\vlan\dotso\vlan B_j|\;,$
and therefore $\pi_i(u')\in|B_j|$, and $\sn(\pi_i(u'))\in|B_j|$.
%
Because $y$ is weakened, we have $\sn(\lambda y.\pi_i(u'))=\lambda y.u''\share*{}y$ with $u''=\sn(\pi_i(u'))$.
%
Since $u''\in|B_j|$, by Lemma~\ref{lem:weakening RED} we have $u''\share*{}v\in|B_j|$ for any $v\in\SN$.
%
Therefore $\lambda y.u''\share*{}y\in|V\vlim B_j|$ for any formula $V$, and in particular $\lambda y.u''\share*{}y\in|A_j\to B_j|$.
%
This means that $\sn(\lambda y.\pi_i(u'))$ is in $|\bigcap_{j\leq n}(A_j\vlim B_j)|$, as is any variant of it.
%
By the induction hypothesis, $s'\subn{x_i}{\sn(\lambda y.\pi_i(u'))^i}\in|C|$ and by Lemma~\ref{lem:Red-AddSharings}
\[
	s'\subn{x_i}{\sn(\lambda y.\pi_i(u'))^i}[\sh(\lambda y.u^n)]\in|C|\;.
\]
It follows by Lemma~\ref{lem:IntCaseLambdaSharing} that $s'\distr xy{u'}\in|C|$.
%
\end{proof}

\noindent{\bf Remark:}
It should be noted that as a consequence of strong normalisation for $D_a$, we get the preservation of the strong normalisation with respect to lambda-calculus (often called PSN property), using the well know fact that the strongly normalisable lambda-terms are typable in $D$ (\cite{Coppo-DezaniCiancaglini-1980,Pottinger-1980,Krivine-1993}).

\begin{ALtheorem}[PSN]
If $N$ is strongly normalisable then $\tercoden N$ is strongly normalisable.
\end{ALtheorem}

\begin{proof}
If $N$ is strongly normalisable, then it is typeable in $D$; then by a generalization of Proposition~\ref{prop:types preserved}, $\tercoden N$ is typeable in $D_a$, and therefore  $\tercoden N$ is strongly normalisable.
\end{proof}

% =================================================================================================
\subsection{Typing of Strongly Normalising Atomic Lambda Terms}

We first show that atomic lambda-terms in normal form are typeable in $D_a$ and then extends the typing to any strongly normalisable  atomic lambda-term, by induction on the sum of the lengths of the reduction paths.

\begin{ALproposition}
\label{prop:intersection weakening}
If $\Gamma, x:A\vdash t:B$ then $\Gamma,x:A\cap C\vdash t:B$.
\end{ALproposition}

% \begin{ALproposition}
% \label{prop:variant typing}
% If $t_1$ and $t_2$ are variants of the same term and $x_1:A_1,\dots,x_n:A_n\vdash_D t_1:B$, then $y_1:A_1,\dots,y_n:A_n\vdash_D t_2:B$.
% \end{ALproposition}
% 
% \begin{ALproposition}
% \label{prop:typing substitution}
% If $\Gamma,\Delta \vdash_D u\sub xt:A$, then there exists a $B$, such that
% $\Gamma, x:B \vdash_D u:A$ and $\Delta\vdash_D t:B$.
% \end{ALproposition}
% 
% \begin{ALdefinition}
% Let $\ALtype$ be the rewrite system $\rightsquigarrow$ where the rule $u\share x{(t)v}\ALnotbeta u\{(y_i)z_i/x_i\}_{i\leq n}\share yt\share zv$ is restricted to $n\geq 1$.
% \end{ALdefinition}
% 
% The aim of this restricted notion of reduction is to be able to reason about reverse preservation of typing:
% 
% NOTE: we need to make clear that in the below $t$ is the redex itself (rather than the redex being a sub-term).
% 
% \begin{ALlemma}\label{lem:reflection}
% If $t\ALtype u$ and $\Gamma \vdash_D u:A$, then there is a $\Gamma'$, such that $\Gamma' \vdash_D t:A$.
% \end{ALlemma}
% 
% \begin{proof}By induction on the length of the reduction path. We examine each reduction rule:
% \begin{itemize}
% %\begin{equation}\tag{$\beta$}\label{eqn:beta-reduction}
% %	(\lambda x.u)t ~\ALbeta~ u\sub xt
% %\end{equation}
% \item
% Consider rule~(\ref{eqn:beta-reduction}): Given $\Gamma,\Delta \vdash_D u\sub xt:A$, we have by Proposition~\ref{prop:typing substitution}, some $B$ such that
% $\Gamma, x:B \vdash_D u:A$ and $\Delta\vdash_D t:B$, which gives $\Gamma,\Delta\vdash_D (\lambda x.u)t$.
% %\label{eqn:sharing above abstraction}
% %	\lambda x.t\g & ~\ALnotbeta~ (\lambda x.t)\g
% %		 & \text{if}~x\in\FV(t) &&&
% \item
% Consider rule~(\ref{eqn:sharing above abstraction}). There are two cases:
% \begin{itemize}
% \item
% Given $\Gamma,\Delta \vdash_D (\lambda x.t)\share*{\vv*y}v:A$, we have $\Gamma,x:B,\vv*y:C\vdash_D t:A$ and
% $\Delta \vdash_D v:C$, which gives $\Gamma,\Delta\vdash_D \lambda x.(t\share*{\vv*y}v):A$.
% \item
% Given $\Gamma,\Delta \vdash_D (\lambda x.t)\distr*{\vv*y}zv:A$, we have $\Gamma,x:B,\vv*y:\bigcap_{i\leq n}(C_i\vlim D_i)\vdash_D t:A$ and
% for each $i\leq n$, $\Delta,z:C^\star_i \vdash_D v:D_i$, which gives $\Gamma,\Delta\vdash_D \lambda x.(t\distr*{\vv*y}zv):A$.
% \end{itemize}
% %\label{eqn:sharing above application function}
% %	(u\g)t & ~\ALnotbeta~ ((u)t)\g
% \item
% Consider rule~(\ref{eqn:sharing above application function}). There are two cases:
% \begin{itemize}
% \item
% Given $\Gamma,\Delta,\Sigma \vdash_D ((u)t)\share*{\vv*y}v:A$. As this is a redex of the given rule $\vv*y$ binds only in $u$ and
% we have $\Gamma,\vv*y:C\vdash_D u:B\vlim A$, $\Delta\vdash_D t:B$ and
% $\Sigma \vdash_D v:C$, which gives $\Gamma,\Delta,\Sigma\vdash_D (u\share*{\vv*y}v)t:A$.
% \item
% Given $\Gamma,\Delta,\Sigma \vdash_D ((u)t)\distr*{\vv*y}zv:A$. As this is a redex of the given rule $\vv*y$ binds only in $u$ and
% we have $\Gamma,\vv*y:\bigcap_{i\leq n}(C_i\vlim D_i)\vdash_D u:B\vlim A$, $\Delta\vdash_D t:B$ and,
% for each $i\leq n$, $\Sigma,z:C^\star_i \vdash_D v:D_i$, which gives $\Gamma,\Delta,\Sigma\vdash_D (u\distr*{\vv*y}zv)t:A$.
% \end{itemize}
% %\label{eqn:sharing above application argument}
% %	(u)t\g & ~\ALnotbeta~ ((u)t)\g
% \item
% Consider rule~(\ref{eqn:sharing above application argument}). There are two cases:
% \begin{itemize}
% \item
% Given $\Gamma,\Delta,\Sigma \vdash_D ((u)t)\share*{\vv*y}v:A$. As this is a redex of the given rule $\vv*y$ binds only in $t$ and
% we have $\Gamma\vdash_D u:B\vlim A$, $\Delta,\vv*y:C\vdash_D t:B$ and
% $\Sigma \vdash_D v:C$, which gives $\Gamma,\Delta,\Sigma\vdash_D (u)t\share*{\vv*y}v:A$.
% \item
% Given $\Gamma,\Delta,\Sigma \vdash_D ((u)t)\distr*{\vv*y}zv:A$. As this is a redex of the given rule $\vv*y$ binds only in $t$ and
% we have $\Gamma \vdash_D u:B\vlim A$, $\Delta,\vv*y:\bigcap_{i\leq n}(C_i\vlim D_i)\vdash_D t:B$ and,
% for each $i\leq n$, $\Sigma,z:C^\star_i \vdash_D v:D_i$, which gives $\Gamma,\Delta,\Sigma\vdash_D (u)t\distr*{\vv*y}zv):A$.
% \end{itemize}
% %\label{eqn:sharing above sharing}
% %	u\share x{t\g} & ~\ALnotbeta~ u\share xt\g
% \item
% Consider rule~(\ref{eqn:sharing above sharing}). There are two cases:
% \begin{itemize}
% \item
% Given $\Gamma,\Delta,\Sigma^+ \vdash_D u\share*{\vv*x}t\share*{\vv*y}v:A$. As this is a redex of the given rule $\vv*y$ binds only in $t$ and
% we have $\Gamma,\vv*x:B\vdash_D u:A$, $\Delta,\vv*y:C\vdash_D t:B$ and
% $\Sigma^+ \vdash_D v:C$. In case $u$ is an $n$-term and $\vv*y$ is the empty sequence, we may have $\Sigma^+\neq\Sigma$,
% in which case we also have $\Sigma\vdash_D v:C$,
% which gives $\Gamma,\Delta,\Sigma\vdash_D u\share*{\vv*x}{t\share*{\vv*y}v}:A$.
% \item
% Given $\Gamma,\Delta,\Sigma \vdash_D u\share*{\vv*x}t\distr*{\vv*y}zv:A$. As this is a redex of the given rule $\vv*y$ binds only in $t$ and
% we have $\Gamma,\vv*x:B \vdash_D u:A$, $\Delta,\vv*y:\bigcap_{i\leq n}(C_i\vlim D_i)\vdash_D t:B$ and,
% for each $i\leq n$, $\Sigma,z:C^\star_i \vdash_D v:D_i$, which gives $\Gamma,\Delta,\Sigma\vdash_D u\share*{\vv*x}{t\distr*{\vv*y}zv}:A$.
% \end{itemize}
% %\label{eqn:sharing above distribution}
% %	u\distr xy{t\g} & ~\ALnotbeta~ u\distr xyt\g
% %		& \text{if}~y\in\FV(t) &&&
% \item
% Consider rule~(\ref{eqn:sharing above distribution}). There are two cases:
% \begin{itemize}
% \item
% Given $\Gamma,\Delta,\Sigma^+ \vdash_D u\distr*{\vv*x}qt\share*{\vv*y}v:A$. As this is a redex of the given rule $\vv*y$ binds only in $t$ and
% we have $\Gamma,\vv*x:\bigcap_{i\leq n}(E_i\vlim B_i)\vdash_D u:A$, for each $i\leq n$, $\Delta,\vv*y:C,q:E^\star_i\vdash_D t:B_i$ and
% $\Sigma^+ \vdash_D v:C$. In case $u$ is an $n$-term and $\vv*y$ is the empty sequence, we may have $\Sigma^+\neq\Sigma$,
% in which case we also have $\Sigma\vdash_D v:C$,
% which gives $\Gamma,\Delta,\Sigma\vdash_D u\distr*{\vv*x}q{t\share*{\vv*y}v}:A$.
% \item
% Given $\Gamma,\Delta,\Sigma \vdash_D u\distr*{\vv*x}qt\distr*{\vv*y}zv:A$. As this is a redex of the given rule $\vv*y$ binds only in $t$ and
% we have $\Gamma,\vv*x:\bigcap_{i\leq n}(E_i\vlim B_i) \vdash_D u:A$, for each $i\leq n$, $\Delta,\vv*y:\bigcap_{j\leq m}(C_j\vlim D_j),q:E^\star_i\vdash_D t:B$ and,
% for each $j\leq m$, $\Sigma,z:C^\star_j \vdash_D v:D_j$, which gives $\Gamma,\Delta,\Sigma\vdash_D u\distr*{\vv*x}q{t\distr*{\vv*y}zv}:A$.
% \end{itemize}
% %\label{eqn:execute unary substitution}
% %	u\share*xt~\ALnotbeta~u\sub xt
% \item
% Consider rule~(\ref{eqn:execute unary substitution}): Given $\Gamma,\Delta \vdash_D u\sub xt:A$, we have by Proposition~\ref{prop:typing substitution}, some $B$ such that
% $\Gamma, x:B \vdash_D u:A$ and $\Delta\vdash_D t:B$, which gives $\Gamma,\Delta\vdash_D u\share*xt$.
% %\label{eqn:compound sharings}
% %	u\share*{\vv*x}{y_i}\share yt ~\ALnotbeta~
% %	u\share*{y_1,\dots,y_{i-1},\vv*x,y_{i+1},\dots,y_n}{t}
% \item
% Consider rule~(\ref{eqn:compound sharings}):
% Given $\Gamma,\Delta\vdash_D u\share*{y_1,\dots,y_{i-1},\vv*x,y_{i+1},\dots,y_n}{t}:A$, we have $\Gamma,y_1,\dots,y_{i-1},\vv*x,y_{i+1},\dots,y_n:B\vdash_D u:A$ and
% $\Delta\vdash_D t:B$. This gives $\Gamma,\vv*y:B\vdash_D u\share*{\vv*x}{y_i}$, which again gives $\Gamma,\Delta\vdash_D u\share*{\vv*x}{y_i}\share*{\vv*y}t$.
% %\label{eqn:share application}
% %	u\share x{(v)t} ~\ALnotbeta~
% %	u\subn{x_i}{(y_i)z_i}\share yv\share zt
% \item
% Consider rule~(\ref{eqn:share application}), for $n\geq1$:
% Given $\Gamma,\Delta,\Sigma \vdash_D u\subn{x_i}{(y_i)z_i}\share yv\share zt:A$. As this is a redex of the given rule, $\vv*z$ binds only in $u$,
% we have $\Gamma,\vv*y:D,\vv*z:C\vdash_D u\subn{x_i}{(y_i)z_i}:A$;
% and by Proposition~\ref{prop:typing substitution}, there is an $B$, such that $D=C\vlim B$.
% We then have $\Delta\vdash_D v:C\vlim B$ and $\Sigma\vdash_D t:C$, which gives $\Gamma,\Delta,\Sigma\vdash_D u\share*{\vv*x}{(v)t}:A$.
% %\label{eqn:share abstraction}
% %	u\share x{\lambda x.t}~\ALnotbeta~ u\distr xx{\tup y\share yt}
% \item
% Consider rule~(\ref{eqn:share abstraction}):
% Given $\Gamma,\Delta\vdash_D u\distr*{\vv*x}y{\tup z\share zt}:A$, we have $\Gamma,\vv*x:\bigcap_{i\leq n}(B_i\vlim C_i)\vdash_D u:A$ and,
% for each $i\leq n$, $\Delta,y:B^\star_i\vdash_D t:C_i$, which gives
% $\Gamma,\Delta\vdash_D u\share*{\vv*x}{\lambda y.t}:A$. Note that in case $n=0$, $B$, may be a fake type $B^+$, but this does not affect the argument.
% %\label{eqn:distributor elimination}
% %	u\distr xy{\tup t\share*{\vv*z}{y}}~\ALnotbeta~
% %	u\subn{x_i}{\lambda y_i.t_i\share*{\vv*z_i}{y_i}}
% \item
% Consider rule~(\ref{eqn:distributor elimination}): Given
% \[
% \Gamma,\Delta_1,\dots,\Delta_n\vdash_D u\subn{x_i}{\lambda y_i.t_i\share*{\vv*z_i}{y_i}}:A\;,
% \]
% with each $t_i$ a fresh variant of the same term $t$, we have by Proposition~\ref{prop:typing substitution}, some $B_i\vlim C_i$'s such that
% \[
% \Gamma, (x_i:B_i\vlim C_i)_{i\leq n}\vdash_D u:A\;;
% \]
% and for each $i\leq n$,
% \[
% \Delta_i\vdash_D\lambda y_i.t_i\share*{\vv*z_i}{y_i}:B_i\vlim C_i\;.
% \]
% By Proposition~\ref{prop:intersection weakening} we get
% \[
% \Gamma, (x_i:\bigcap_{i\leq n}(B_i\vlim C_i))_{i\leq n}\vdash_D u:A\;,
% \]
% and by Proposition~\ref{prop:variant typing} and~\ref{prop:intersection weakening} we get
% for each $i,j\leq n$, $\Delta_i',\vv*z_i:B_j\vdash_D t_i:C_j$, where the $i^{\mbox{th}}$ variable in $\Delta_i'$ is $i^{\mbox{th}}$ variable of $\Delta_i$ with the type being the intersection of each of the types of the $i^{\mbox{th}}$ variable of each of the $\Delta_j$'s. Put together, we get, for each $j\leq n$,
% \[
% \Delta_1',\dots,\Delta_n',y:B_j\vdash_D \tup t\share*{\vv*z}{y}:\vls(C_j.\dots.C_j)\;.
% \]
% We then get
% \[
% 	\Gamma,\Delta'_1,\dots,\Delta'_n\vdash_D u\distr xy{\tup t\share*{\vv*z}{y}}: A\;.
% \]
% \end{itemize}
% \end{proof}
% 
\begin{ALproposition}
\label{prop:normal form}
If a term is normal, it is of one of the following forms, where $u$ and each $u_i$ are normal.
\[
	x \qquad \lambda x.u \qquad u\share xy \qquad (\dotso(x)u_1\dotso )u_n
\]
\end{ALproposition}

\begin{ALproposition}
\label{prop:typable normal form}
For an atomic lambda-term $t$ in normal form there exist $\Gamma$, $A$ such that $\Gamma\vdash_{D_a} t:A$.
\end{ALproposition}

\begin{proof}
The proof proceeds by induction on $t$, following Proposition~\ref{prop:normal form}.
%
\begin{enumerate}[1)]

	\item
Let $t=x$. Then $x:A\vdash x:A$ for any $A$.

	\item
Let $t=\lambda x.u$ with $u$ normal.
%
By the induction hypothesis there are $\Gamma$, $A$, and $B$ such that $\Gamma, x:A\vdash u:B$.
%
Then $\Gamma\vdash\lambda x.u:A\vlim B$.

	\item
let $t=u\share xy$ with $u$ normal.
%
By the induction hypothesis there are $\Gamma$, $A$, and $B_i$ for $i\leq n$ such that $\Gamma,(x_i:B_i)_{i\leq n}\vdash u:A$.
%
Let $B=\bigcap_{i\leq n}B_i$.
%
By Proposition~\ref{prop:intersection weakening} $\Gamma,(x_i:B)_{i\leq n}\vdash u:A$, and by the inference rule $(\typeshare)$, for sharing, $\Gamma\vdash u\share xy:A$.

	\item
Let $t=(\dotso(x)u_1\dotso )u_n$ with each $u_i$ normal.
%
By the induction hypothesis there are $\Gamma_i$ and $A_i$ for each $i\leq n$ such that $\Gamma_i \vdash u_i: A_i$.
%
For any $B$, the variable $x$ is typed by $x: A_1 \vlim \dotso \vlim A_n \vlim B \vdash x: A_1 \vlim \dotso \vlim A_n \vlim B$.
%
Then $(\Gamma_i)_{i\leq n},x: A_1 \vlim \dotso \vlim A_n \vlim B \vdash (\dotso(x)u_1\dotso )u_n: B$.

\end{enumerate}
\end{proof}

\begin{ALtheorem}
If $u$ is strongly normalising, then it is typeable with intersection types.
\end{ALtheorem}

\begin{proof}
By induction on $S(u)$. One consider the possible forms of terms and choose in each case a specific reduction which allows to to reverse the type from the reduct to the original term. The following example shows why the relaxing of the type contraints in the rules $\typeshare_{D_a}$ and $\typedist_{D_a}$ to type all the strongly normalisable atomic $\lambda$-terms.
\\
\textcolor{red}{Tom: can you provide one or two nice examples with a sentence of explanation?}


% \begin{itemize}
% \item
% If $S(u)=0$, the result follows by Proposition~\ref{prop:normal form}.
% \item
% If $u$ is a redex of $\ALtype$, there is a $v$ such that $u\ALtype v$. Then $S(v)<S(u)$ so $v$ is typable by the induction hypothesis and $u$ is typable by Lemma~\ref{lem:reflection}.
% \item
% If $u=w\share*{}{(t)v}$, both $w,(t)v\in\SN$ and $S(w),S((t)v)<S(u)$, so by the induction hypothesis we have $\Gamma\vdash_D w:A$ and $\Delta\vdash_D (t)v:B$, which gives $\Gamma,\Delta\vdash_D w\share*{}{(t)v}:A$.
% \item
% If $u=w\distr*{\vv*x}y{s^n\share*{}{(t)v}}$, both
% $w\distr*{\vv*x}y{s^n\share*{}y},(t)v\in\SN$
% and
% $S(w\distr*{\vv*x}y{s^n\share*{}y}),S((t)v)<S(u)$,
% so by the induction hypothesis we have $\Gamma,\vv*x:\bigcap_{i\leq m}(B_i\vlim C_i)\vdash_D w:A$ and, for each $i\leq m$, $\Delta\vdash_D s^n:C_i$, and $\Sigma,y:X\vdash_D (t)v:Y$. We then get, for each $i\leq m$, $\Delta,y:X^+\vdash_D s^n\share*{}{(t)v}:C_i$, which gives $\Gamma,\Delta,\Sigma\vdash_D w\distr*{\vv*x}y{s^n\share*{}{(t)v}}:A$.
% %the induction hypothesis we have $\Gamma, (x_i:B_i\vlim C_i)_{i\leq n}\vdash_D v:A$ and $\Delta,y:X\vdash_D w:B$ so we have $\Delta,y:X^+\vdash_D w:B$ and $\Gamma,\Delta\vdash_D v\distr xy{t^n\share*{}w}:A$.
% \item
% If $u$ is not a redex, procede by induction on the structure of $u$:
% \begin{itemize}
% \item
% $u=\lambda x.t$, $t$ is typeable by the induction hypothesis, and $u$ is typable as in the proof of Proposition~\ref{prop:typable normal form}.
% \item
% $u=(\dotso(x)t_1\dotso )t_n$, by the induction hypothesis each of $t_1$, $\dots$, $t_n$ and so is $u$ as in the proof of Proposition~\ref{prop:typable normal form}.
% \item
% $u=v\share xy$, by the induction hypothesis we have $\Gamma,(x_i:B_i)_{i\leq n}\vdash_D v:A$ so $\Gamma,\Delta\vdash_D v\share xy:A$ as in the proof of Proposition~\ref{prop:typable normal form}.
% \end{itemize}
% \end{itemize}
\end{proof}


% =================================================================================================
\section{Conclusions and further work}



The present result, of strong normalisation for the atomic lambda-calculus with intersection types, emphasises how the calculus is a natural and well-behaved formalisation of sharing in the lambda-calculus.
%
Future investigations will expand in three directions: strengthening the current strong normalisation result; adapting the atomic lambda-calculus to address further notions of sharing; and investigating the practical use of the calculus in computation, for instance in compiling or implementing functional programming languages.



The present work trongly suggests an agle for future research: it is expected that the type system and strong normalisation proof can be extended to the second-order case---although subject reduction is not immediately obvious in this case.
%, where type inference (in e.g.\ a  Hindley--Milner type system) becomes useful


For the atomic lambda-calculus in general, further work will focus on variations on the calculus that more closely approach the reduction dynamics of sharing graphs, to encompass further degrees of sharing.
%
Another direction would be the inclusion of general recursion in the calculus, and the investigation of its interaction with the sharing constructs, as a prerequisite of making the calculus useful in practice to the implementation of functional programming languages.


